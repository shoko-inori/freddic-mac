\documentclass[11pt]{article}

    \usepackage[breakable]{tcolorbox}
    \usepackage{parskip} % Stop auto-indenting (to mimic markdown behaviour)
    
    \usepackage{iftex}
    \ifPDFTeX
    	\usepackage[T1]{fontenc}
    	\usepackage{mathpazo}
    \else
    	\usepackage{fontspec}
    \fi

    % Basic figure setup, for now with no caption control since it's done
    % automatically by Pandoc (which extracts ![](path) syntax from Markdown).
    \usepackage{graphicx}
    % Maintain compatibility with old templates. Remove in nbconvert 6.0
    \let\Oldincludegraphics\includegraphics
    % Ensure that by default, figures have no caption (until we provide a
    % proper Figure object with a Caption API and a way to capture that
    % in the conversion process - todo).
    \usepackage{caption}
    \DeclareCaptionFormat{nocaption}{}
    \captionsetup{format=nocaption,aboveskip=0pt,belowskip=0pt}

    \usepackage[Export]{adjustbox} % Used to constrain images to a maximum size
    \adjustboxset{max size={0.9\linewidth}{0.9\paperheight}}
    \usepackage{float}
    \floatplacement{figure}{H} % forces figures to be placed at the correct location
    \usepackage{xcolor} % Allow colors to be defined
    \usepackage{enumerate} % Needed for markdown enumerations to work
    \usepackage{geometry} % Used to adjust the document margins
    \usepackage{amsmath} % Equations
    \usepackage{amssymb} % Equations
    \usepackage{textcomp} % defines textquotesingle
    % Hack from http://tex.stackexchange.com/a/47451/13684:
    \AtBeginDocument{%
        \def\PYZsq{\textquotesingle}% Upright quotes in Pygmentized code
    }
    \usepackage{upquote} % Upright quotes for verbatim code
    \usepackage{eurosym} % defines \euro
    \usepackage[mathletters]{ucs} % Extended unicode (utf-8) support
    \usepackage{fancyvrb} % verbatim replacement that allows latex
    \usepackage{grffile} % extends the file name processing of package graphics 
                         % to support a larger range
    \makeatletter % fix for grffile with XeLaTeX
    \def\Gread@@xetex#1{%
      \IfFileExists{"\Gin@base".bb}%
      {\Gread@eps{\Gin@base.bb}}%
      {\Gread@@xetex@aux#1}%
    }
    \makeatother

    % The hyperref package gives us a pdf with properly built
    % internal navigation ('pdf bookmarks' for the table of contents,
    % internal cross-reference links, web links for URLs, etc.)
    \usepackage{hyperref}
    % The default LaTeX title has an obnoxious amount of whitespace. By default,
    % titling removes some of it. It also provides customization options.
    \usepackage{titling}
    \usepackage{longtable} % longtable support required by pandoc >1.10
    \usepackage{booktabs}  % table support for pandoc > 1.12.2
    \usepackage[inline]{enumitem} % IRkernel/repr support (it uses the enumerate* environment)
    \usepackage[normalem]{ulem} % ulem is needed to support strikethroughs (\sout)
                                % normalem makes italics be italics, not underlines
    \usepackage{mathrsfs}
    

    
    % Colors for the hyperref package
    \definecolor{urlcolor}{rgb}{0,.145,.698}
    \definecolor{linkcolor}{rgb}{.71,0.21,0.01}
    \definecolor{citecolor}{rgb}{.12,.54,.11}

    % ANSI colors
    \definecolor{ansi-black}{HTML}{3E424D}
    \definecolor{ansi-black-intense}{HTML}{282C36}
    \definecolor{ansi-red}{HTML}{E75C58}
    \definecolor{ansi-red-intense}{HTML}{B22B31}
    \definecolor{ansi-green}{HTML}{00A250}
    \definecolor{ansi-green-intense}{HTML}{007427}
    \definecolor{ansi-yellow}{HTML}{DDB62B}
    \definecolor{ansi-yellow-intense}{HTML}{B27D12}
    \definecolor{ansi-blue}{HTML}{208FFB}
    \definecolor{ansi-blue-intense}{HTML}{0065CA}
    \definecolor{ansi-magenta}{HTML}{D160C4}
    \definecolor{ansi-magenta-intense}{HTML}{A03196}
    \definecolor{ansi-cyan}{HTML}{60C6C8}
    \definecolor{ansi-cyan-intense}{HTML}{258F8F}
    \definecolor{ansi-white}{HTML}{C5C1B4}
    \definecolor{ansi-white-intense}{HTML}{A1A6B2}
    \definecolor{ansi-default-inverse-fg}{HTML}{FFFFFF}
    \definecolor{ansi-default-inverse-bg}{HTML}{000000}

    % commands and environments needed by pandoc snippets
    % extracted from the output of `pandoc -s`
    \providecommand{\tightlist}{%
      \setlength{\itemsep}{0pt}\setlength{\parskip}{0pt}}
    \DefineVerbatimEnvironment{Highlighting}{Verbatim}{commandchars=\\\{\}}
    % Add ',fontsize=\small' for more characters per line
    \newenvironment{Shaded}{}{}
    \newcommand{\KeywordTok}[1]{\textcolor[rgb]{0.00,0.44,0.13}{\textbf{{#1}}}}
    \newcommand{\DataTypeTok}[1]{\textcolor[rgb]{0.56,0.13,0.00}{{#1}}}
    \newcommand{\DecValTok}[1]{\textcolor[rgb]{0.25,0.63,0.44}{{#1}}}
    \newcommand{\BaseNTok}[1]{\textcolor[rgb]{0.25,0.63,0.44}{{#1}}}
    \newcommand{\FloatTok}[1]{\textcolor[rgb]{0.25,0.63,0.44}{{#1}}}
    \newcommand{\CharTok}[1]{\textcolor[rgb]{0.25,0.44,0.63}{{#1}}}
    \newcommand{\StringTok}[1]{\textcolor[rgb]{0.25,0.44,0.63}{{#1}}}
    \newcommand{\CommentTok}[1]{\textcolor[rgb]{0.38,0.63,0.69}{\textit{{#1}}}}
    \newcommand{\OtherTok}[1]{\textcolor[rgb]{0.00,0.44,0.13}{{#1}}}
    \newcommand{\AlertTok}[1]{\textcolor[rgb]{1.00,0.00,0.00}{\textbf{{#1}}}}
    \newcommand{\FunctionTok}[1]{\textcolor[rgb]{0.02,0.16,0.49}{{#1}}}
    \newcommand{\RegionMarkerTok}[1]{{#1}}
    \newcommand{\ErrorTok}[1]{\textcolor[rgb]{1.00,0.00,0.00}{\textbf{{#1}}}}
    \newcommand{\NormalTok}[1]{{#1}}
    
    % Additional commands for more recent versions of Pandoc
    \newcommand{\ConstantTok}[1]{\textcolor[rgb]{0.53,0.00,0.00}{{#1}}}
    \newcommand{\SpecialCharTok}[1]{\textcolor[rgb]{0.25,0.44,0.63}{{#1}}}
    \newcommand{\VerbatimStringTok}[1]{\textcolor[rgb]{0.25,0.44,0.63}{{#1}}}
    \newcommand{\SpecialStringTok}[1]{\textcolor[rgb]{0.73,0.40,0.53}{{#1}}}
    \newcommand{\ImportTok}[1]{{#1}}
    \newcommand{\DocumentationTok}[1]{\textcolor[rgb]{0.73,0.13,0.13}{\textit{{#1}}}}
    \newcommand{\AnnotationTok}[1]{\textcolor[rgb]{0.38,0.63,0.69}{\textbf{\textit{{#1}}}}}
    \newcommand{\CommentVarTok}[1]{\textcolor[rgb]{0.38,0.63,0.69}{\textbf{\textit{{#1}}}}}
    \newcommand{\VariableTok}[1]{\textcolor[rgb]{0.10,0.09,0.49}{{#1}}}
    \newcommand{\ControlFlowTok}[1]{\textcolor[rgb]{0.00,0.44,0.13}{\textbf{{#1}}}}
    \newcommand{\OperatorTok}[1]{\textcolor[rgb]{0.40,0.40,0.40}{{#1}}}
    \newcommand{\BuiltInTok}[1]{{#1}}
    \newcommand{\ExtensionTok}[1]{{#1}}
    \newcommand{\PreprocessorTok}[1]{\textcolor[rgb]{0.74,0.48,0.00}{{#1}}}
    \newcommand{\AttributeTok}[1]{\textcolor[rgb]{0.49,0.56,0.16}{{#1}}}
    \newcommand{\InformationTok}[1]{\textcolor[rgb]{0.38,0.63,0.69}{\textbf{\textit{{#1}}}}}
    \newcommand{\WarningTok}[1]{\textcolor[rgb]{0.38,0.63,0.69}{\textbf{\textit{{#1}}}}}
    
    
    % Define a nice break command that doesn't care if a line doesn't already
    % exist.
    \def\br{\hspace*{\fill} \\* }
    % Math Jax compatibility definitions
    \def\gt{>}
    \def\lt{<}
    \let\Oldtex\TeX
    \let\Oldlatex\LaTeX
    \renewcommand{\TeX}{\textrm{\Oldtex}}
    \renewcommand{\LaTeX}{\textrm{\Oldlatex}}
    % Document parameters
    \title{Analysis on the Single Family Loan-Level Dataset by Freddie Mac}
    \author{Chenhao Gong}
    \date{February 2020}
    
    
    
    
    
% Pygments definitions
\makeatletter
\def\PY@reset{\let\PY@it=\relax \let\PY@bf=\relax%
    \let\PY@ul=\relax \let\PY@tc=\relax%
    \let\PY@bc=\relax \let\PY@ff=\relax}
\def\PY@tok#1{\csname PY@tok@#1\endcsname}
\def\PY@toks#1+{\ifx\relax#1\empty\else%
    \PY@tok{#1}\expandafter\PY@toks\fi}
\def\PY@do#1{\PY@bc{\PY@tc{\PY@ul{%
    \PY@it{\PY@bf{\PY@ff{#1}}}}}}}
\def\PY#1#2{\PY@reset\PY@toks#1+\relax+\PY@do{#2}}

\expandafter\def\csname PY@tok@w\endcsname{\def\PY@tc##1{\textcolor[rgb]{0.73,0.73,0.73}{##1}}}
\expandafter\def\csname PY@tok@c\endcsname{\let\PY@it=\textit\def\PY@tc##1{\textcolor[rgb]{0.25,0.50,0.50}{##1}}}
\expandafter\def\csname PY@tok@cp\endcsname{\def\PY@tc##1{\textcolor[rgb]{0.74,0.48,0.00}{##1}}}
\expandafter\def\csname PY@tok@k\endcsname{\let\PY@bf=\textbf\def\PY@tc##1{\textcolor[rgb]{0.00,0.50,0.00}{##1}}}
\expandafter\def\csname PY@tok@kp\endcsname{\def\PY@tc##1{\textcolor[rgb]{0.00,0.50,0.00}{##1}}}
\expandafter\def\csname PY@tok@kt\endcsname{\def\PY@tc##1{\textcolor[rgb]{0.69,0.00,0.25}{##1}}}
\expandafter\def\csname PY@tok@o\endcsname{\def\PY@tc##1{\textcolor[rgb]{0.40,0.40,0.40}{##1}}}
\expandafter\def\csname PY@tok@ow\endcsname{\let\PY@bf=\textbf\def\PY@tc##1{\textcolor[rgb]{0.67,0.13,1.00}{##1}}}
\expandafter\def\csname PY@tok@nb\endcsname{\def\PY@tc##1{\textcolor[rgb]{0.00,0.50,0.00}{##1}}}
\expandafter\def\csname PY@tok@nf\endcsname{\def\PY@tc##1{\textcolor[rgb]{0.00,0.00,1.00}{##1}}}
\expandafter\def\csname PY@tok@nc\endcsname{\let\PY@bf=\textbf\def\PY@tc##1{\textcolor[rgb]{0.00,0.00,1.00}{##1}}}
\expandafter\def\csname PY@tok@nn\endcsname{\let\PY@bf=\textbf\def\PY@tc##1{\textcolor[rgb]{0.00,0.00,1.00}{##1}}}
\expandafter\def\csname PY@tok@ne\endcsname{\let\PY@bf=\textbf\def\PY@tc##1{\textcolor[rgb]{0.82,0.25,0.23}{##1}}}
\expandafter\def\csname PY@tok@nv\endcsname{\def\PY@tc##1{\textcolor[rgb]{0.10,0.09,0.49}{##1}}}
\expandafter\def\csname PY@tok@no\endcsname{\def\PY@tc##1{\textcolor[rgb]{0.53,0.00,0.00}{##1}}}
\expandafter\def\csname PY@tok@nl\endcsname{\def\PY@tc##1{\textcolor[rgb]{0.63,0.63,0.00}{##1}}}
\expandafter\def\csname PY@tok@ni\endcsname{\let\PY@bf=\textbf\def\PY@tc##1{\textcolor[rgb]{0.60,0.60,0.60}{##1}}}
\expandafter\def\csname PY@tok@na\endcsname{\def\PY@tc##1{\textcolor[rgb]{0.49,0.56,0.16}{##1}}}
\expandafter\def\csname PY@tok@nt\endcsname{\let\PY@bf=\textbf\def\PY@tc##1{\textcolor[rgb]{0.00,0.50,0.00}{##1}}}
\expandafter\def\csname PY@tok@nd\endcsname{\def\PY@tc##1{\textcolor[rgb]{0.67,0.13,1.00}{##1}}}
\expandafter\def\csname PY@tok@s\endcsname{\def\PY@tc##1{\textcolor[rgb]{0.73,0.13,0.13}{##1}}}
\expandafter\def\csname PY@tok@sd\endcsname{\let\PY@it=\textit\def\PY@tc##1{\textcolor[rgb]{0.73,0.13,0.13}{##1}}}
\expandafter\def\csname PY@tok@si\endcsname{\let\PY@bf=\textbf\def\PY@tc##1{\textcolor[rgb]{0.73,0.40,0.53}{##1}}}
\expandafter\def\csname PY@tok@se\endcsname{\let\PY@bf=\textbf\def\PY@tc##1{\textcolor[rgb]{0.73,0.40,0.13}{##1}}}
\expandafter\def\csname PY@tok@sr\endcsname{\def\PY@tc##1{\textcolor[rgb]{0.73,0.40,0.53}{##1}}}
\expandafter\def\csname PY@tok@ss\endcsname{\def\PY@tc##1{\textcolor[rgb]{0.10,0.09,0.49}{##1}}}
\expandafter\def\csname PY@tok@sx\endcsname{\def\PY@tc##1{\textcolor[rgb]{0.00,0.50,0.00}{##1}}}
\expandafter\def\csname PY@tok@m\endcsname{\def\PY@tc##1{\textcolor[rgb]{0.40,0.40,0.40}{##1}}}
\expandafter\def\csname PY@tok@gh\endcsname{\let\PY@bf=\textbf\def\PY@tc##1{\textcolor[rgb]{0.00,0.00,0.50}{##1}}}
\expandafter\def\csname PY@tok@gu\endcsname{\let\PY@bf=\textbf\def\PY@tc##1{\textcolor[rgb]{0.50,0.00,0.50}{##1}}}
\expandafter\def\csname PY@tok@gd\endcsname{\def\PY@tc##1{\textcolor[rgb]{0.63,0.00,0.00}{##1}}}
\expandafter\def\csname PY@tok@gi\endcsname{\def\PY@tc##1{\textcolor[rgb]{0.00,0.63,0.00}{##1}}}
\expandafter\def\csname PY@tok@gr\endcsname{\def\PY@tc##1{\textcolor[rgb]{1.00,0.00,0.00}{##1}}}
\expandafter\def\csname PY@tok@ge\endcsname{\let\PY@it=\textit}
\expandafter\def\csname PY@tok@gs\endcsname{\let\PY@bf=\textbf}
\expandafter\def\csname PY@tok@gp\endcsname{\let\PY@bf=\textbf\def\PY@tc##1{\textcolor[rgb]{0.00,0.00,0.50}{##1}}}
\expandafter\def\csname PY@tok@go\endcsname{\def\PY@tc##1{\textcolor[rgb]{0.53,0.53,0.53}{##1}}}
\expandafter\def\csname PY@tok@gt\endcsname{\def\PY@tc##1{\textcolor[rgb]{0.00,0.27,0.87}{##1}}}
\expandafter\def\csname PY@tok@err\endcsname{\def\PY@bc##1{\setlength{\fboxsep}{0pt}\fcolorbox[rgb]{1.00,0.00,0.00}{1,1,1}{\strut ##1}}}
\expandafter\def\csname PY@tok@kc\endcsname{\let\PY@bf=\textbf\def\PY@tc##1{\textcolor[rgb]{0.00,0.50,0.00}{##1}}}
\expandafter\def\csname PY@tok@kd\endcsname{\let\PY@bf=\textbf\def\PY@tc##1{\textcolor[rgb]{0.00,0.50,0.00}{##1}}}
\expandafter\def\csname PY@tok@kn\endcsname{\let\PY@bf=\textbf\def\PY@tc##1{\textcolor[rgb]{0.00,0.50,0.00}{##1}}}
\expandafter\def\csname PY@tok@kr\endcsname{\let\PY@bf=\textbf\def\PY@tc##1{\textcolor[rgb]{0.00,0.50,0.00}{##1}}}
\expandafter\def\csname PY@tok@bp\endcsname{\def\PY@tc##1{\textcolor[rgb]{0.00,0.50,0.00}{##1}}}
\expandafter\def\csname PY@tok@fm\endcsname{\def\PY@tc##1{\textcolor[rgb]{0.00,0.00,1.00}{##1}}}
\expandafter\def\csname PY@tok@vc\endcsname{\def\PY@tc##1{\textcolor[rgb]{0.10,0.09,0.49}{##1}}}
\expandafter\def\csname PY@tok@vg\endcsname{\def\PY@tc##1{\textcolor[rgb]{0.10,0.09,0.49}{##1}}}
\expandafter\def\csname PY@tok@vi\endcsname{\def\PY@tc##1{\textcolor[rgb]{0.10,0.09,0.49}{##1}}}
\expandafter\def\csname PY@tok@vm\endcsname{\def\PY@tc##1{\textcolor[rgb]{0.10,0.09,0.49}{##1}}}
\expandafter\def\csname PY@tok@sa\endcsname{\def\PY@tc##1{\textcolor[rgb]{0.73,0.13,0.13}{##1}}}
\expandafter\def\csname PY@tok@sb\endcsname{\def\PY@tc##1{\textcolor[rgb]{0.73,0.13,0.13}{##1}}}
\expandafter\def\csname PY@tok@sc\endcsname{\def\PY@tc##1{\textcolor[rgb]{0.73,0.13,0.13}{##1}}}
\expandafter\def\csname PY@tok@dl\endcsname{\def\PY@tc##1{\textcolor[rgb]{0.73,0.13,0.13}{##1}}}
\expandafter\def\csname PY@tok@s2\endcsname{\def\PY@tc##1{\textcolor[rgb]{0.73,0.13,0.13}{##1}}}
\expandafter\def\csname PY@tok@sh\endcsname{\def\PY@tc##1{\textcolor[rgb]{0.73,0.13,0.13}{##1}}}
\expandafter\def\csname PY@tok@s1\endcsname{\def\PY@tc##1{\textcolor[rgb]{0.73,0.13,0.13}{##1}}}
\expandafter\def\csname PY@tok@mb\endcsname{\def\PY@tc##1{\textcolor[rgb]{0.40,0.40,0.40}{##1}}}
\expandafter\def\csname PY@tok@mf\endcsname{\def\PY@tc##1{\textcolor[rgb]{0.40,0.40,0.40}{##1}}}
\expandafter\def\csname PY@tok@mh\endcsname{\def\PY@tc##1{\textcolor[rgb]{0.40,0.40,0.40}{##1}}}
\expandafter\def\csname PY@tok@mi\endcsname{\def\PY@tc##1{\textcolor[rgb]{0.40,0.40,0.40}{##1}}}
\expandafter\def\csname PY@tok@il\endcsname{\def\PY@tc##1{\textcolor[rgb]{0.40,0.40,0.40}{##1}}}
\expandafter\def\csname PY@tok@mo\endcsname{\def\PY@tc##1{\textcolor[rgb]{0.40,0.40,0.40}{##1}}}
\expandafter\def\csname PY@tok@ch\endcsname{\let\PY@it=\textit\def\PY@tc##1{\textcolor[rgb]{0.25,0.50,0.50}{##1}}}
\expandafter\def\csname PY@tok@cm\endcsname{\let\PY@it=\textit\def\PY@tc##1{\textcolor[rgb]{0.25,0.50,0.50}{##1}}}
\expandafter\def\csname PY@tok@cpf\endcsname{\let\PY@it=\textit\def\PY@tc##1{\textcolor[rgb]{0.25,0.50,0.50}{##1}}}
\expandafter\def\csname PY@tok@c1\endcsname{\let\PY@it=\textit\def\PY@tc##1{\textcolor[rgb]{0.25,0.50,0.50}{##1}}}
\expandafter\def\csname PY@tok@cs\endcsname{\let\PY@it=\textit\def\PY@tc##1{\textcolor[rgb]{0.25,0.50,0.50}{##1}}}

\def\PYZbs{\char`\\}
\def\PYZus{\char`\_}
\def\PYZob{\char`\{}
\def\PYZcb{\char`\}}
\def\PYZca{\char`\^}
\def\PYZam{\char`\&}
\def\PYZlt{\char`\<}
\def\PYZgt{\char`\>}
\def\PYZsh{\char`\#}
\def\PYZpc{\char`\%}
\def\PYZdl{\char`\$}
\def\PYZhy{\char`\-}
\def\PYZsq{\char`\'}
\def\PYZdq{\char`\"}
\def\PYZti{\char`\~}
% for compatibility with earlier versions
\def\PYZat{@}
\def\PYZlb{[}
\def\PYZrb{]}
\makeatother


    % For linebreaks inside Verbatim environment from package fancyvrb. 
    \makeatletter
        \newbox\Wrappedcontinuationbox 
        \newbox\Wrappedvisiblespacebox 
        \newcommand*\Wrappedvisiblespace {\textcolor{red}{\textvisiblespace}} 
        \newcommand*\Wrappedcontinuationsymbol {\textcolor{red}{\llap{\tiny$\m@th\hookrightarrow$}}} 
        \newcommand*\Wrappedcontinuationindent {3ex } 
        \newcommand*\Wrappedafterbreak {\kern\Wrappedcontinuationindent\copy\Wrappedcontinuationbox} 
        % Take advantage of the already applied Pygments mark-up to insert 
        % potential linebreaks for TeX processing. 
        %        {, <, #, %, $, ' and ": go to next line. 
        %        _, }, ^, &, >, - and ~: stay at end of broken line. 
        % Use of \textquotesingle for straight quote. 
        \newcommand*\Wrappedbreaksatspecials {% 
            \def\PYGZus{\discretionary{\char`\_}{\Wrappedafterbreak}{\char`\_}}% 
            \def\PYGZob{\discretionary{}{\Wrappedafterbreak\char`\{}{\char`\{}}% 
            \def\PYGZcb{\discretionary{\char`\}}{\Wrappedafterbreak}{\char`\}}}% 
            \def\PYGZca{\discretionary{\char`\^}{\Wrappedafterbreak}{\char`\^}}% 
            \def\PYGZam{\discretionary{\char`\&}{\Wrappedafterbreak}{\char`\&}}% 
            \def\PYGZlt{\discretionary{}{\Wrappedafterbreak\char`\<}{\char`\<}}% 
            \def\PYGZgt{\discretionary{\char`\>}{\Wrappedafterbreak}{\char`\>}}% 
            \def\PYGZsh{\discretionary{}{\Wrappedafterbreak\char`\#}{\char`\#}}% 
            \def\PYGZpc{\discretionary{}{\Wrappedafterbreak\char`\%}{\char`\%}}% 
            \def\PYGZdl{\discretionary{}{\Wrappedafterbreak\char`\$}{\char`\$}}% 
            \def\PYGZhy{\discretionary{\char`\-}{\Wrappedafterbreak}{\char`\-}}% 
            \def\PYGZsq{\discretionary{}{\Wrappedafterbreak\textquotesingle}{\textquotesingle}}% 
            \def\PYGZdq{\discretionary{}{\Wrappedafterbreak\char`\"}{\char`\"}}% 
            \def\PYGZti{\discretionary{\char`\~}{\Wrappedafterbreak}{\char`\~}}% 
        } 
        % Some characters . , ; ? ! / are not pygmentized. 
        % This macro makes them "active" and they will insert potential linebreaks 
        \newcommand*\Wrappedbreaksatpunct {% 
            \lccode`\~`\.\lowercase{\def~}{\discretionary{\hbox{\char`\.}}{\Wrappedafterbreak}{\hbox{\char`\.}}}% 
            \lccode`\~`\,\lowercase{\def~}{\discretionary{\hbox{\char`\,}}{\Wrappedafterbreak}{\hbox{\char`\,}}}% 
            \lccode`\~`\;\lowercase{\def~}{\discretionary{\hbox{\char`\;}}{\Wrappedafterbreak}{\hbox{\char`\;}}}% 
            \lccode`\~`\:\lowercase{\def~}{\discretionary{\hbox{\char`\:}}{\Wrappedafterbreak}{\hbox{\char`\:}}}% 
            \lccode`\~`\?\lowercase{\def~}{\discretionary{\hbox{\char`\?}}{\Wrappedafterbreak}{\hbox{\char`\?}}}% 
            \lccode`\~`\!\lowercase{\def~}{\discretionary{\hbox{\char`\!}}{\Wrappedafterbreak}{\hbox{\char`\!}}}% 
            \lccode`\~`\/\lowercase{\def~}{\discretionary{\hbox{\char`\/}}{\Wrappedafterbreak}{\hbox{\char`\/}}}% 
            \catcode`\.\active
            \catcode`\,\active 
            \catcode`\;\active
            \catcode`\:\active
            \catcode`\?\active
            \catcode`\!\active
            \catcode`\/\active 
            \lccode`\~`\~ 	
        }
    \makeatother

    \let\OriginalVerbatim=\Verbatim
    \makeatletter
    \renewcommand{\Verbatim}[1][1]{%
        %\parskip\z@skip
        \sbox\Wrappedcontinuationbox {\Wrappedcontinuationsymbol}%
        \sbox\Wrappedvisiblespacebox {\FV@SetupFont\Wrappedvisiblespace}%
        \def\FancyVerbFormatLine ##1{\hsize\linewidth
            \vtop{\raggedright\hyphenpenalty\z@\exhyphenpenalty\z@
                \doublehyphendemerits\z@\finalhyphendemerits\z@
                \strut ##1\strut}%
        }%
        % If the linebreak is at a space, the latter will be displayed as visible
        % space at end of first line, and a continuation symbol starts next line.
        % Stretch/shrink are however usually zero for typewriter font.
        \def\FV@Space {%
            \nobreak\hskip\z@ plus\fontdimen3\font minus\fontdimen4\font
            \discretionary{\copy\Wrappedvisiblespacebox}{\Wrappedafterbreak}
            {\kern\fontdimen2\font}%
        }%
        
        % Allow breaks at special characters using \PYG... macros.
        \Wrappedbreaksatspecials
        % Breaks at punctuation characters . , ; ? ! and / need catcode=\active 	
        \OriginalVerbatim[#1,codes*=\Wrappedbreaksatpunct]%
    }
    \makeatother

    % Exact colors from NB
    \definecolor{incolor}{HTML}{303F9F}
    \definecolor{outcolor}{HTML}{D84315}
    \definecolor{cellborder}{HTML}{CFCFCF}
    \definecolor{cellbackground}{HTML}{F7F7F7}
    
    % prompt
    \makeatletter
    \newcommand{\boxspacing}{\kern\kvtcb@left@rule\kern\kvtcb@boxsep}
    \makeatother
    \newcommand{\prompt}[4]{
        \ttfamily\llap{{\color{#2}[#3]:\hspace{3pt}#4}}\vspace{-\baselineskip}
    }
    

    
    % Prevent overflowing lines due to hard-to-break entities
    \sloppy 
    % Setup hyperref package
    \hypersetup{
      breaklinks=true,  % so long urls are correctly broken across lines
      colorlinks=true,
      urlcolor=urlcolor,
      linkcolor=linkcolor,
      citecolor=citecolor,
      }
    % Slightly bigger margins than the latex defaults
    
    \geometry{verbose,tmargin=1in,bmargin=1in,lmargin=1in,rmargin=1in}
    
    

\begin{document}
    
    \maketitle
    
    

    
    \hypertarget{introduction}{%
\section{Introduction}\label{introduction}}

In today's North America, mortgage has become a major financial method
homebuyers employ to afford new homes. From a data scientist's
perspective, I am interested in the information that the backgrounds of
mortgage applications can provide us, and also I feel curious about the
factors that may affect the performance of repayment of a mortgage.

In this analysis study, I was using the November 2019 release of the
Single Family Loan-Level Dataset from Freddie Mac, a mortgage loaner in
the United States. The dataset contains the information of the mortgages
originated from January 1, 1999 through September 30, 2018, with monthly
loan performance data through March 31, 2019, that were sold to Freddie
Mac or that back Freddie Mac Participation Certificates (PCs). The
dataset is divided into two sections: the Origination Dataset and the
Monthly Performance Dataset. Throughout the analysis, I will be using
both sets included in the dataset to answer a number of question
regarded to mortgage origination backgrounds as well as payment
performances.

This project was inspired by Senso.ai of Toronto. The firm also contributed to the project by introducing me to the dataset. 

    \begin{tcolorbox}[breakable, size=fbox, boxrule=1pt, pad at break*=1mm,colback=cellbackground, colframe=cellborder]
\prompt{In}{incolor}{1}{\boxspacing}
\begin{Verbatim}[commandchars=\\\{\}]
\PY{c+c1}{\PYZsh{} Import the needed packages}
\PY{k+kn}{import} \PY{n+nn}{numpy} \PY{k}{as} \PY{n+nn}{np}
\PY{k+kn}{import} \PY{n+nn}{pandas} \PY{k}{as} \PY{n+nn}{pd}
\PY{k+kn}{import} \PY{n+nn}{scipy}
\PY{k+kn}{from} \PY{n+nn}{IPython}\PY{n+nn}{.}\PY{n+nn}{display} \PY{k}{import} \PY{n}{Image}
\PY{k+kn}{import} \PY{n+nn}{matplotlib}\PY{n+nn}{.}\PY{n+nn}{pyplot} \PY{k}{as} \PY{n+nn}{plt}
\PY{k+kn}{from} \PY{n+nn}{sklearn}\PY{n+nn}{.}\PY{n+nn}{tree} \PY{k}{import} \PY{n}{DecisionTreeClassifier}\PY{p}{,} \PY{n}{export\PYZus{}graphviz}
\PY{k+kn}{from} \PY{n+nn}{six} \PY{k}{import} \PY{n}{StringIO}
\PY{k+kn}{import} \PY{n+nn}{pydotplus}
\end{Verbatim}
\end{tcolorbox}

    \hypertarget{analysis-on-the-origination-dataset}{%
\section{Analysis on the Origination
Dataset}\label{analysis-on-the-origination-dataset}}

The Origination dataset contains the information of the recorded
mortgages at Freddie Mac, at the time the mortgage was initialized. It
represents mainly the general backgrounds of the mortgages, which may
provide us with (but not limited to) the price and area of the property
secured by the mortgage, and financial background and history of the
individual, and the channels that initialized the purchase and the
mortgage.

    \begin{tcolorbox}[breakable, size=fbox, boxrule=1pt, pad at break*=1mm,colback=cellbackground, colframe=cellborder]
\prompt{In}{incolor}{2}{\boxspacing}
\begin{Verbatim}[commandchars=\\\{\}]
\PY{c+c1}{\PYZsh{} Read the origination samples csv file}
\PY{n}{origination\PYZus{}data} \PY{o}{=} \PY{n}{pd}\PY{o}{.}\PY{n}{read\PYZus{}csv}\PY{p}{(}\PY{l+s+s2}{\PYZdq{}}\PY{l+s+s2}{origination\PYZus{}sample.csv}\PY{l+s+s2}{\PYZdq{}}\PY{p}{)}
\end{Verbatim}
\end{tcolorbox}

    \hypertarget{first-time-homebuyers}{%
\subsection{First-time Homebuyers}\label{first-time-homebuyers}}

In the Origination database, the first-time homebuyers are identified by
a specific flag. In real life, they may be eligible of designated
policies, offers and/or credits, either from the government or financial
institutions. From the mortgage provider's perspective, we want to see
the structure of all homebuyers in the database, and figure out the role
of first-time homebuyers within other buyers.

On the first rough glimpse, I computed the percentage of first-buyers
among all buyers of all times as presented in the database.

    \begin{tcolorbox}[breakable, size=fbox, boxrule=1pt, pad at break*=1mm,colback=cellbackground, colframe=cellborder]
\prompt{In}{incolor}{3}{\boxspacing}
\begin{Verbatim}[commandchars=\\\{\}]
\PY{c+c1}{\PYZsh{} Percentage of first time homebuyers}
\PY{n}{first\PYZus{}time\PYZus{}data} \PY{o}{=} \PY{n}{origination\PYZus{}data}\PY{o}{.}\PY{n}{groupby}\PY{p}{(}\PY{l+s+s2}{\PYZdq{}}\PY{l+s+s2}{first\PYZus{}time\PYZus{}homebuyer\PYZus{}flag}\PY{l+s+s2}{\PYZdq{}}\PY{p}{)}
\PY{n}{first\PYZus{}buyer} \PY{o}{=} \PY{n}{first\PYZus{}time\PYZus{}data}\PY{o}{.}\PY{n}{size}\PY{p}{(}\PY{p}{)}\PY{p}{[}\PY{l+s+s2}{\PYZdq{}}\PY{l+s+s2}{Y}\PY{l+s+s2}{\PYZdq{}}\PY{p}{]}
\PY{n}{non\PYZus{}first\PYZus{}buyer} \PY{o}{=} \PY{n}{first\PYZus{}time\PYZus{}data}\PY{o}{.}\PY{n}{size}\PY{p}{(}\PY{p}{)}\PY{p}{[}\PY{l+s+s2}{\PYZdq{}}\PY{l+s+s2}{N}\PY{l+s+s2}{\PYZdq{}}\PY{p}{]}
\PY{n}{first\PYZus{}buyer} \PY{o}{/} \PY{p}{(}\PY{n}{first\PYZus{}buyer} \PY{o}{+} \PY{n}{non\PYZus{}first\PYZus{}buyer}\PY{p}{)}
\end{Verbatim}
\end{tcolorbox}

            \begin{tcolorbox}[breakable, size=fbox, boxrule=.5pt, pad at break*=1mm, opacityfill=0]
\prompt{Out}{outcolor}{3}{\boxspacing}
\begin{Verbatim}[commandchars=\\\{\}]
0.16807615740394258
\end{Verbatim}
\end{tcolorbox}
        
    As the computation above suggests, it turns out that 16.8076\% of all
homebuyers are categorized as first-time homebuyers at all times
according to the provided database.

To further analyse the number, portion and role of first-time
homebuyers, I made further computations to find the relationship between
portion of first-time homebuyers and the month, while comparing this
relationship to that between the number of all homebuyers and the month.

    \begin{tcolorbox}[breakable, size=fbox, boxrule=1pt, pad at break*=1mm,colback=cellbackground, colframe=cellborder]
\prompt{In}{incolor}{4}{\boxspacing}
\begin{Verbatim}[commandchars=\\\{\}]
\PY{c+c1}{\PYZsh{} Percentage of first time homebuyers in each month}
\PY{n}{new\PYZus{}origination\PYZus{}data} \PY{o}{=} \PY{n}{origination\PYZus{}data}\PY{o}{.}\PY{n}{copy}\PY{p}{(}\PY{p}{)}
\PY{n}{new\PYZus{}origination\PYZus{}data}\PY{p}{[}\PY{l+s+s2}{\PYZdq{}}\PY{l+s+s2}{month}\PY{l+s+s2}{\PYZdq{}}\PY{p}{]} \PY{o}{=} \PY{n}{origination\PYZus{}data}\PY{p}{[}\PY{l+s+s2}{\PYZdq{}}\PY{l+s+s2}{first\PYZus{}payment\PYZus{}date}\PY{l+s+s2}{\PYZdq{}}\PY{p}{]} \PY{o}{\PYZpc{}} \PY{l+m+mi}{100}
\PY{n}{month\PYZus{}firsthome\PYZus{}data} \PY{o}{=} \PY{n}{new\PYZus{}origination\PYZus{}data}\PY{o}{.}\PY{n}{loc}\PY{p}{[}\PY{p}{:}\PY{p}{,} \PY{p}{[}\PY{l+s+s2}{\PYZdq{}}\PY{l+s+s2}{month}\PY{l+s+s2}{\PYZdq{}}\PY{p}{,} \PY{l+s+s2}{\PYZdq{}}\PY{l+s+s2}{first\PYZus{}time\PYZus{}homebuyer\PYZus{}flag}\PY{l+s+s2}{\PYZdq{}}\PY{p}{]}\PY{p}{]}
\PY{n}{month\PYZus{}firsthome\PYZus{}data} \PY{o}{=} \PY{n}{month\PYZus{}firsthome\PYZus{}data}\PY{p}{[}\PY{n}{month\PYZus{}firsthome\PYZus{}data}\PY{p}{[}\PY{l+s+s2}{\PYZdq{}}\PY{l+s+s2}{first\PYZus{}time\PYZus{}homebuyer\PYZus{}flag}\PY{l+s+s2}{\PYZdq{}}\PY{p}{]} \PY{o}{!=} \PY{l+s+s2}{\PYZdq{}}\PY{l+s+s2}{9}\PY{l+s+s2}{\PYZdq{}}\PY{p}{]}

\PY{n}{month\PYZus{}firsthome\PYZus{}count} \PY{o}{=} \PY{n}{month\PYZus{}firsthome\PYZus{}data}\PY{o}{.}\PY{n}{groupby}\PY{p}{(}
    \PY{p}{[}\PY{l+s+s2}{\PYZdq{}}\PY{l+s+s2}{month}\PY{l+s+s2}{\PYZdq{}}\PY{p}{,} \PY{l+s+s2}{\PYZdq{}}\PY{l+s+s2}{first\PYZus{}time\PYZus{}homebuyer\PYZus{}flag}\PY{l+s+s2}{\PYZdq{}}\PY{p}{]}
\PY{p}{)}\PY{o}{.}\PY{n}{size}\PY{p}{(}\PY{p}{)}

\PY{c+c1}{\PYZsh{} Compute ft\PYZus{}monthly\PYZus{}percentage}

\PY{n}{ft\PYZus{}monthly\PYZus{}percentage} \PY{o}{=} \PY{n}{np}\PY{o}{.}\PY{n}{array}\PY{p}{(}\PY{p}{[}\PY{n}{np}\PY{o}{.}\PY{n}{nan}\PY{p}{]} \PY{o}{*} \PY{l+m+mi}{12}\PY{p}{)}
\PY{n}{total\PYZus{}array} \PY{o}{=} \PY{n}{np}\PY{o}{.}\PY{n}{array}\PY{p}{(}\PY{p}{[}\PY{n}{np}\PY{o}{.}\PY{n}{nan}\PY{p}{]} \PY{o}{*} \PY{l+m+mi}{12}\PY{p}{)}

\PY{n}{total\PYZus{}first\PYZus{}time} \PY{o}{=} \PY{n}{np}\PY{o}{.}\PY{n}{array}\PY{p}{(}\PY{p}{[}\PY{n}{np}\PY{o}{.}\PY{n}{nan}\PY{p}{]} \PY{o}{*} \PY{l+m+mi}{12}\PY{p}{)}

\PY{k}{for} \PY{n}{month} \PY{o+ow}{in} \PY{n+nb}{range}\PY{p}{(}\PY{l+m+mi}{1}\PY{p}{,} \PY{l+m+mi}{13}\PY{p}{)}\PY{p}{:} 
    \PY{n}{first\PYZus{}buyer} \PY{o}{=} \PY{n}{month\PYZus{}firsthome\PYZus{}count}\PY{p}{[}\PY{n}{month}\PY{p}{]}\PY{p}{[}\PY{l+s+s2}{\PYZdq{}}\PY{l+s+s2}{Y}\PY{l+s+s2}{\PYZdq{}}\PY{p}{]}
    \PY{n}{non\PYZus{}first\PYZus{}buyer} \PY{o}{=} \PY{n}{month\PYZus{}firsthome\PYZus{}count}\PY{p}{[}\PY{n}{month}\PY{p}{]}\PY{p}{[}\PY{l+s+s2}{\PYZdq{}}\PY{l+s+s2}{N}\PY{l+s+s2}{\PYZdq{}}\PY{p}{]}
    \PY{n}{total\PYZus{}array}\PY{p}{[}\PY{n}{month} \PY{o}{\PYZhy{}} \PY{l+m+mi}{1}\PY{p}{]} \PY{o}{=} \PY{n}{first\PYZus{}buyer} \PY{o}{+} \PY{n}{non\PYZus{}first\PYZus{}buyer}
    \PY{n}{ft\PYZus{}monthly\PYZus{}percentage}\PY{p}{[}\PY{n}{month} \PY{o}{\PYZhy{}} \PY{l+m+mi}{1}\PY{p}{]} \PY{o}{=} \PY{n}{first\PYZus{}buyer} \PY{o}{/} \PY{n}{total\PYZus{}array}\PY{p}{[}\PY{n}{month} \PY{o}{\PYZhy{}} \PY{l+m+mi}{1}\PY{p}{]}
    \PY{n}{total\PYZus{}first\PYZus{}time}\PY{p}{[}\PY{n}{month} \PY{o}{\PYZhy{}} \PY{l+m+mi}{1}\PY{p}{]} \PY{o}{=} \PY{n}{first\PYZus{}buyer}

\PY{c+c1}{\PYZsh{} Compute ft\PYZus{}annual\PYZus{}percentage }
\PY{n}{ft\PYZus{}annual\PYZus{}percentage} \PY{o}{=} \PY{n}{np}\PY{o}{.}\PY{n}{array}\PY{p}{(}
    \PY{p}{[}\PY{n}{month} \PY{o}{/} \PY{n+nb}{sum}\PY{p}{(}\PY{n}{total\PYZus{}first\PYZus{}time}\PY{p}{)} \PY{k}{for} \PY{n}{month} \PY{o+ow}{in} \PY{n}{total\PYZus{}first\PYZus{}time}\PY{p}{]}
\PY{p}{)}

\PY{c+c1}{\PYZsh{} Compute homebuyer\PYZus{}percentage}
\PY{n}{homebuyer\PYZus{}percentage} \PY{o}{=} \PY{n}{np}\PY{o}{.}\PY{n}{array}\PY{p}{(}
    \PY{p}{[}\PY{n}{month} \PY{o}{/} \PY{n+nb}{sum}\PY{p}{(}\PY{n}{total\PYZus{}array}\PY{p}{)} \PY{k}{for} \PY{n}{month} \PY{o+ow}{in} \PY{n}{total\PYZus{}array}\PY{p}{]}
\PY{p}{)}

\PY{n}{percentage\PYZus{}table} \PY{o}{=} \PY{n}{pd}\PY{o}{.}\PY{n}{DataFrame}\PY{p}{(}\PY{p}{\PYZob{}}
    \PY{l+s+s2}{\PYZdq{}}\PY{l+s+s2}{month}\PY{l+s+s2}{\PYZdq{}}\PY{p}{:} \PY{p}{[}\PY{n}{month} \PY{k}{for} \PY{n}{month} \PY{o+ow}{in} \PY{n+nb}{range}\PY{p}{(}\PY{l+m+mi}{1}\PY{p}{,} \PY{l+m+mi}{13}\PY{p}{)}\PY{p}{]}\PY{p}{,} 
    \PY{l+s+s2}{\PYZdq{}}\PY{l+s+s2}{total\PYZus{}first\PYZus{}time}\PY{l+s+s2}{\PYZdq{}}\PY{p}{:} \PY{n}{total\PYZus{}first\PYZus{}time}\PY{p}{,}
    \PY{l+s+s2}{\PYZdq{}}\PY{l+s+s2}{total\PYZus{}buyers}\PY{l+s+s2}{\PYZdq{}}\PY{p}{:} \PY{n}{total\PYZus{}array}\PY{p}{,}
    \PY{l+s+s2}{\PYZdq{}}\PY{l+s+s2}{ft\PYZus{}monthly\PYZus{}percentage}\PY{l+s+s2}{\PYZdq{}}\PY{p}{:} \PY{n}{ft\PYZus{}monthly\PYZus{}percentage}\PY{p}{,}
    \PY{l+s+s2}{\PYZdq{}}\PY{l+s+s2}{ft\PYZus{}annual\PYZus{}percentage}\PY{l+s+s2}{\PYZdq{}}\PY{p}{:} \PY{n}{ft\PYZus{}annual\PYZus{}percentage}\PY{p}{,}
    \PY{l+s+s2}{\PYZdq{}}\PY{l+s+s2}{homebuyer\PYZus{}percentage}\PY{l+s+s2}{\PYZdq{}}\PY{p}{:} \PY{n}{homebuyer\PYZus{}percentage}\PY{p}{,}
\PY{p}{\PYZcb{}}\PY{p}{)}
\PY{n}{percentage\PYZus{}table}
\end{Verbatim}
\end{tcolorbox}

            \begin{tcolorbox}[breakable, size=fbox, boxrule=.5pt, pad at break*=1mm, opacityfill=0]
\prompt{Out}{outcolor}{4}{\boxspacing}
\begin{Verbatim}[commandchars=\\\{\}]
    month  total\_first\_time  total\_buyers  ft\_monthly\_percentage  \textbackslash{}
0       1             849.0        5066.0               0.167588
1       2             862.0        4811.0               0.179173
2       3             579.0        4457.0               0.129908
3       4             702.0        5086.0               0.138026
4       5             812.0        5895.0               0.137744
5       6            1000.0        5922.0               0.168862
6       7            1095.0        6085.0               0.179951
7       8            1228.0        6463.0               0.190005
8       9            1148.0        6169.0               0.186092
9      10            1069.0        5824.0               0.183551
10     11             969.0        5407.0               0.179212
11     12             916.0        5624.0               0.162873

    ft\_annual\_percentage  homebuyer\_percentage
0               0.075608              0.075828
1               0.076766              0.072011
2               0.051563              0.066713
3               0.062517              0.076127
4               0.072313              0.088237
5               0.089055              0.088641
6               0.097515              0.091081
7               0.109360              0.096738
8               0.102235              0.092338
9               0.095200              0.087174
10              0.086294              0.080932
11              0.081574              0.084180
\end{Verbatim}
\end{tcolorbox}
        
    In the data frame above, I listed the numbers of first-time homebuyers
and all homebuyers occurring in each of the 12 months. With the data, I
computed for each month:

\begin{itemize}
\tightlist
\item
  \texttt{ft\_monthly\_percentage}, the percentage of first-time
  homebuyers among all homebuyers within the month,
\item
  \texttt{ft\_annual\_percentage}, the percentage of first-time
  homebuyers occuring in the particular month among all first-time
  homebuyers recorded in the dataset, and
\item
  \texttt{homebuyer\_percentage}, the percentage of homebuyers
  (including both first-time and non-first-time ones) occuring in the
  particular month among all homebuyers recorded in the dataset
\end{itemize}

To benefit the interpretation of the computed percentages, I visualized
them in a scatterplot, as shown below:

    \begin{tcolorbox}[breakable, size=fbox, boxrule=1pt, pad at break*=1mm,colback=cellbackground, colframe=cellborder]
\prompt{In}{incolor}{5}{\boxspacing}
\begin{Verbatim}[commandchars=\\\{\}]
\PY{c+c1}{\PYZsh{} Plot between first\PYZus{}time\PYZus{}homebuyer\PYZus{}percentage and percentage\PYZus{}annual}
\PY{n}{month\PYZus{}array} \PY{o}{=} \PY{p}{[}\PY{n}{month} \PY{k}{for} \PY{n}{month} \PY{o+ow}{in} \PY{n+nb}{range}\PY{p}{(}\PY{l+m+mi}{1}\PY{p}{,} \PY{l+m+mi}{13}\PY{p}{)}\PY{p}{]}

\PY{n}{plt}\PY{o}{.}\PY{n}{plot}\PY{p}{(}\PY{n}{month\PYZus{}array}\PY{p}{,} \PY{n}{ft\PYZus{}monthly\PYZus{}percentage}\PY{p}{,} \PY{l+s+s1}{\PYZsq{}}\PY{l+s+s1}{g*}\PY{l+s+s1}{\PYZsq{}}\PY{p}{,} \PY{n}{label}\PY{o}{=}\PY{l+s+s2}{\PYZdq{}}\PY{l+s+s2}{ft\PYZus{}monthly\PYZus{}percentage}\PY{l+s+s2}{\PYZdq{}}\PY{p}{)}
\PY{n}{plt}\PY{o}{.}\PY{n}{plot}\PY{p}{(}\PY{n}{month\PYZus{}array}\PY{p}{,} \PY{n}{ft\PYZus{}annual\PYZus{}percentage}\PY{p}{,} \PY{l+s+s1}{\PYZsq{}}\PY{l+s+s1}{rs}\PY{l+s+s1}{\PYZsq{}}\PY{p}{,} \PY{n}{label}\PY{o}{=}\PY{l+s+s2}{\PYZdq{}}\PY{l+s+s2}{ft\PYZus{}annual\PYZus{}percentage}\PY{l+s+s2}{\PYZdq{}}\PY{p}{)}
\PY{n}{plt}\PY{o}{.}\PY{n}{plot}\PY{p}{(}\PY{n}{month\PYZus{}array}\PY{p}{,} \PY{n}{homebuyer\PYZus{}percentage}\PY{p}{,} \PY{l+s+s1}{\PYZsq{}}\PY{l+s+s1}{bs}\PY{l+s+s1}{\PYZsq{}}\PY{p}{,} \PY{n}{label}\PY{o}{=}\PY{l+s+s2}{\PYZdq{}}\PY{l+s+s2}{homebuyer\PYZus{}percentage}\PY{l+s+s2}{\PYZdq{}}\PY{p}{)}

\PY{n}{plt}\PY{o}{.}\PY{n}{legend}\PY{p}{(}\PY{n}{bbox\PYZus{}to\PYZus{}anchor}\PY{o}{=}\PY{p}{(}\PY{l+m+mi}{1}\PY{p}{,} \PY{l+m+mi}{1}\PY{p}{)}\PY{p}{)}
\PY{n}{plt}\PY{o}{.}\PY{n}{show}\PY{p}{(}\PY{p}{)}
\end{Verbatim}
\end{tcolorbox}

    \begin{center}
    \adjustimage{max size={0.9\linewidth}{0.9\paperheight}}{report_files/report_9_0.png}
    \end{center}
    { \hspace*{\fill} \\}
    
    From the data, we may find that the peak for first-time homebuyers
within a year is August, while the minimum is observed in March. These
also matches the observations for general homebuyers. However, the
internal monthly distribution also tends to find the same pattern. Hence
we may conclude that while all homebuyers perfer summer and fall seasons
for home purchases than winter and spring, the tendency appears to be
more obvious and dramatic for first-time homebuyers.

A possible interpretation of the observation to the real life may be
that the period from late spring to summer (May to June) sees the climb
of home purchases, and thus becomes preferably the best season of the
year for real estate related businesses to attract new homebuyers that
will likely make their first home purchase.

    \hypertarget{metropolitan-statistical-areas-msa-with-high-original-unpaid-pricipal-balances-upb}{%
\subsection{Metropolitan Statistical Areas (MSA) with High Original
Unpaid Pricipal Balances
(UPB)}\label{metropolitan-statistical-areas-msa-with-high-original-unpaid-pricipal-balances-upb}}

As the dataset represents mortgage records in the United States, the
geographical locations are designated to Metropolitan Statistical Areas
(MSA) defined by the United States Office of Management and Budget
(OMB).

In this part, we will look into the 10 MSA's which have the highest
median original Unpaid Principal Balance (UPB) upon the opening of the
mortgage, and attempt to find explain the nature of their high pricipal
balances.

    \begin{tcolorbox}[breakable, size=fbox, boxrule=1pt, pad at break*=1mm,colback=cellbackground, colframe=cellborder]
\prompt{In}{incolor}{6}{\boxspacing}
\begin{Verbatim}[commandchars=\\\{\}]
\PY{c+c1}{\PYZsh{} MSA\PYZsq{}s with the highest median original unpaid principal balance}
\PY{n}{new\PYZus{}origination\PYZus{}data2} \PY{o}{=} \PY{n}{origination\PYZus{}data}\PY{o}{.}\PY{n}{copy}\PY{p}{(}\PY{p}{)}

\PY{n}{msa\PYZus{}upb\PYZus{}data} \PY{o}{=} \PY{n}{new\PYZus{}origination\PYZus{}data2}\PY{o}{.}\PY{n}{loc}\PY{p}{[}\PY{p}{:}\PY{p}{,} \PY{p}{[}\PY{l+s+s1}{\PYZsq{}}\PY{l+s+s1}{msa}\PY{l+s+s1}{\PYZsq{}}\PY{p}{,} \PY{l+s+s1}{\PYZsq{}}\PY{l+s+s1}{original\PYZus{}upb}\PY{l+s+s1}{\PYZsq{}}\PY{p}{,} \PY{l+s+s1}{\PYZsq{}}\PY{l+s+s1}{original\PYZus{}ltv}\PY{l+s+s1}{\PYZsq{}}\PY{p}{]}\PY{p}{]}
\PY{n}{upb\PYZus{}data} \PY{o}{=} \PY{n}{pd}\PY{o}{.}\PY{n}{DataFrame}\PY{p}{(}
    \PY{n}{msa\PYZus{}upb\PYZus{}data}\PY{o}{.}\PY{n}{groupby}\PY{p}{(}\PY{l+s+s1}{\PYZsq{}}\PY{l+s+s1}{msa}\PY{l+s+s1}{\PYZsq{}}\PY{p}{)}\PY{o}{.}\PY{n}{median}\PY{p}{(}\PY{p}{)}
\PY{p}{)}
\PY{n}{highest\PYZus{}upb} \PY{o}{=} \PY{n}{upb\PYZus{}data}\PY{o}{.}\PY{n}{sort\PYZus{}values}\PY{p}{(}\PY{n}{by}\PY{o}{=}\PY{l+s+s1}{\PYZsq{}}\PY{l+s+s1}{original\PYZus{}upb}\PY{l+s+s1}{\PYZsq{}}\PY{p}{,} \PY{n}{ascending}\PY{o}{=}\PY{k+kc}{False}\PY{p}{)}\PY{o}{.}\PY{n}{head}\PY{p}{(}\PY{n}{n}\PY{o}{=}\PY{l+m+mi}{10}\PY{p}{)}

\PY{n}{highest\PYZus{}upb}\PY{p}{[}\PY{l+s+s1}{\PYZsq{}}\PY{l+s+s1}{original\PYZus{}ltv\PYZus{}percentile}\PY{l+s+s1}{\PYZsq{}}\PY{p}{]} \PY{o}{=} \PY{n}{np}\PY{o}{.}\PY{n}{array}\PY{p}{(}
    \PY{p}{[}\PY{n}{scipy}\PY{o}{.}\PY{n}{stats}\PY{o}{.}\PY{n}{percentileofscore}\PY{p}{(}\PY{n}{msa\PYZus{}upb\PYZus{}data}\PY{o}{.}\PY{n}{loc}\PY{p}{[}\PY{p}{:}\PY{p}{,} \PY{l+s+s1}{\PYZsq{}}\PY{l+s+s1}{original\PYZus{}ltv}\PY{l+s+s1}{\PYZsq{}}\PY{p}{]}\PY{p}{,} \PY{n}{ltv}\PY{p}{)} 
     \PY{k}{for} \PY{n}{ltv} \PY{o+ow}{in} \PY{n}{np}\PY{o}{.}\PY{n}{array}\PY{p}{(}\PY{n}{highest\PYZus{}upb}\PY{p}{[}\PY{l+s+s1}{\PYZsq{}}\PY{l+s+s1}{original\PYZus{}ltv}\PY{l+s+s1}{\PYZsq{}}\PY{p}{]}\PY{p}{)}
    \PY{p}{]}
\PY{p}{)}
    
\PY{n}{highest\PYZus{}upb}
\end{Verbatim}
\end{tcolorbox}

            \begin{tcolorbox}[breakable, size=fbox, boxrule=.5pt, pad at break*=1mm, opacityfill=0]
\prompt{Out}{outcolor}{6}{\boxspacing}
\begin{Verbatim}[commandchars=\\\{\}]
         original\_upb  original\_ltv  original\_ltv\_percentile
msa
42034.0      414000.0          59.0                20.636676
11244.0      364000.0          65.0                27.771634
27980.0      347000.0          54.0                15.795062
42200.0      338000.0          57.0                18.634464
41884.0      330000.0          50.0                12.425749
41940.0      324000.0          55.0                16.709073
46520.0      316000.0          66.0                29.159286
43524.0      288000.0          72.5                40.033468
36084.0      285000.0          64.0                26.400129
42100.0      282000.0          55.5                17.156781
\end{Verbatim}
\end{tcolorbox}
        
    The data frame above shows the codes for the 10 MSA's with the highest
original UPB's, as well as the UPB's exact values. However, the reason
that could lead to the high UPB's in these areas is not determined.

There are two possible assumption to explain the high original UPB
values: it is possible that the price of homes in these areas are higher
and the buyers are generally richer, but it is also possible that the
high ratio of debt for buyers in these regions lead to the high
balances. To determine the true nature, we will be interested in the the
original Loan-To-Value (LTV) included in the dataset, which expresses
the ratio of the mortgaged portion versus the total value of the
property to be purchased.

In the data frame above, I included the original LTV values for each of
the 10 MSA's, as well as their percentile of the original LTV values of
all MSA's that appeared in the dataset.

For the information, below computes the quartiles of original LTV's for
all MSA's appeared in the dataset and shows the data in a data frame.

    \begin{tcolorbox}[breakable, size=fbox, boxrule=1pt, pad at break*=1mm,colback=cellbackground, colframe=cellborder]
\prompt{In}{incolor}{7}{\boxspacing}
\begin{Verbatim}[commandchars=\\\{\}]
\PY{n}{ltv\PYZus{}quartiles} \PY{o}{=} \PY{n}{pd}\PY{o}{.}\PY{n}{DataFrame}\PY{p}{(}
    \PY{n}{np}\PY{o}{.}\PY{n}{quantile}\PY{p}{(}\PY{n}{msa\PYZus{}upb\PYZus{}data}\PY{o}{.}\PY{n}{loc}\PY{p}{[}\PY{p}{:}\PY{p}{,} \PY{l+s+s1}{\PYZsq{}}\PY{l+s+s1}{original\PYZus{}ltv}\PY{l+s+s1}{\PYZsq{}}\PY{p}{]}\PY{p}{,} \PY{p}{[}\PY{l+m+mf}{0.25}\PY{p}{,} \PY{l+m+mf}{0.5}\PY{p}{,} \PY{l+m+mf}{0.75}\PY{p}{]}\PY{p}{)}\PY{o}{.}\PY{n}{reshape}\PY{p}{(}\PY{p}{(}\PY{l+m+mi}{1}\PY{p}{,} \PY{l+m+mi}{3}\PY{p}{)}\PY{p}{)}
\PY{p}{)}
\PY{n}{ltv\PYZus{}quartiles} \PY{o}{=} \PY{n}{ltv\PYZus{}quartiles}\PY{o}{.}\PY{n}{rename}\PY{p}{(}
    \PY{n}{index}\PY{o}{=}\PY{p}{\PYZob{}}\PY{l+m+mi}{0}\PY{p}{:} \PY{l+s+s2}{\PYZdq{}}\PY{l+s+s2}{original\PYZus{}ltv}\PY{l+s+s2}{\PYZdq{}}\PY{p}{\PYZcb{}}\PY{p}{,} 
    \PY{n}{columns}\PY{o}{=}\PY{p}{\PYZob{}}\PY{l+m+mi}{0}\PY{p}{:} \PY{l+s+s2}{\PYZdq{}}\PY{l+s+s2}{first\PYZus{}quartile}\PY{l+s+s2}{\PYZdq{}}\PY{p}{,} \PY{l+m+mi}{1}\PY{p}{:} \PY{l+s+s2}{\PYZdq{}}\PY{l+s+s2}{median}\PY{l+s+s2}{\PYZdq{}}\PY{p}{,} \PY{l+m+mi}{2}\PY{p}{:} \PY{l+s+s2}{\PYZdq{}}\PY{l+s+s2}{third\PYZus{}quartile}\PY{l+s+s2}{\PYZdq{}}\PY{p}{\PYZcb{}}
\PY{p}{)}
\PY{n}{ltv\PYZus{}quartiles}
\end{Verbatim}
\end{tcolorbox}

            \begin{tcolorbox}[breakable, size=fbox, boxrule=.5pt, pad at break*=1mm, opacityfill=0]
\prompt{Out}{outcolor}{7}{\boxspacing}
\begin{Verbatim}[commandchars=\\\{\}]
              first\_quartile  median  third\_quartile
original\_ltv            63.0    76.0            80.0
\end{Verbatim}
\end{tcolorbox}
        
    We can observe that out of the 10 MSA's with the highest original UPB
values, all of them see an original LTV value significantly lower than
the global median, while 9 out of them are near or below the first
quartile. Meanwhile, all of them see a low percentile in the global LTV
values for all MSA's.

The observations favor the first assumption of the two we raised earlier
in this part, indicating that the homes in these areas are generally
higher, and the homebuyers making the purchases are generally richer and
possess more solid financial background. In real practice, one can
convince that the homebuyers purchasing properties in these areas may be
eligible to better rates and/or offers, amd may also have less risk to
default a mortgage.

    \hypertarget{distribution-of-channel}{%
\subsection{Distribution of Channel}\label{distribution-of-channel}}

In the provided dataset, ``channel'' refers to the origin of the home
purchase. The channel of a origination record, if provided and specified
in the dataset, can be any of a broker, a correspondent, or a retail. In
this part of the analysis, I want to find the structure of the
distribution of the channels, and to find the most popular way
homebuyers tend to purchase a new home.

In the computations below, I categorized all records provided in the
dataset, and included the number of occurrances of each of the channel
options, as well as their proportion in the distribution, in a data
frame.

    \begin{tcolorbox}[breakable, size=fbox, boxrule=1pt, pad at break*=1mm,colback=cellbackground, colframe=cellborder]
\prompt{In}{incolor}{8}{\boxspacing}
\begin{Verbatim}[commandchars=\\\{\}]
\PY{c+c1}{\PYZsh{} The distribution of the channel originations in the last 3 years}
\PY{n}{recent\PYZus{}origination\PYZus{}data} \PY{o}{=} \PY{n}{origination\PYZus{}data}\PY{p}{[}\PY{n}{origination\PYZus{}data}\PY{p}{[}\PY{l+s+s2}{\PYZdq{}}\PY{l+s+s2}{first\PYZus{}payment\PYZus{}date}\PY{l+s+s2}{\PYZdq{}}\PY{p}{]} \PY{o}{/}\PY{o}{/} \PY{l+m+mi}{100} \PY{o}{\PYZgt{}}\PY{o}{=} \PY{l+m+mi}{2017}\PY{p}{]}
\PY{n}{recent\PYZus{}channel\PYZus{}dist} \PY{o}{=} \PY{n}{pd}\PY{o}{.}\PY{n}{DataFrame}\PY{p}{(}\PY{n}{recent\PYZus{}origination\PYZus{}data}\PY{o}{.}\PY{n}{groupby}\PY{p}{(}\PY{l+s+s2}{\PYZdq{}}\PY{l+s+s2}{channel}\PY{l+s+s2}{\PYZdq{}}\PY{p}{)}\PY{o}{.}\PY{n}{size}\PY{p}{(}\PY{p}{)}\PY{p}{)}
\PY{n}{recent\PYZus{}channel\PYZus{}dist} \PY{o}{=} \PY{n}{recent\PYZus{}channel\PYZus{}dist}\PY{o}{.}\PY{n}{rename}\PY{p}{(}\PY{n}{columns}\PY{o}{=}\PY{p}{\PYZob{}}\PY{l+m+mi}{0}\PY{p}{:} \PY{l+s+s2}{\PYZdq{}}\PY{l+s+s2}{number}\PY{l+s+s2}{\PYZdq{}}\PY{p}{\PYZcb{}}\PY{p}{)}
\PY{n}{recent\PYZus{}channel\PYZus{}dist} \PY{o}{=} \PY{n}{recent\PYZus{}channel\PYZus{}dist}\PY{o}{.}\PY{n}{rename}\PY{p}{(}\PY{n}{index}\PY{o}{=}\PY{p}{\PYZob{}}\PY{l+s+s2}{\PYZdq{}}\PY{l+s+s2}{B}\PY{l+s+s2}{\PYZdq{}}\PY{p}{:} \PY{l+s+s2}{\PYZdq{}}\PY{l+s+s2}{Broker}\PY{l+s+s2}{\PYZdq{}}\PY{p}{,} \PY{l+s+s2}{\PYZdq{}}\PY{l+s+s2}{C}\PY{l+s+s2}{\PYZdq{}}\PY{p}{:} \PY{l+s+s2}{\PYZdq{}}\PY{l+s+s2}{Correspondent}\PY{l+s+s2}{\PYZdq{}}\PY{p}{,} \PY{l+s+s2}{\PYZdq{}}\PY{l+s+s2}{R}\PY{l+s+s2}{\PYZdq{}}\PY{p}{:} \PY{l+s+s2}{\PYZdq{}}\PY{l+s+s2}{Retail}\PY{l+s+s2}{\PYZdq{}}\PY{p}{\PYZcb{}}\PY{p}{)}
\PY{n}{recent\PYZus{}channel\PYZus{}dist}\PY{p}{[}\PY{l+s+s2}{\PYZdq{}}\PY{l+s+s2}{proportion}\PY{l+s+s2}{\PYZdq{}}\PY{p}{]} \PY{o}{=} \PY{n}{recent\PYZus{}channel\PYZus{}dist}\PY{p}{[}\PY{l+s+s2}{\PYZdq{}}\PY{l+s+s2}{number}\PY{l+s+s2}{\PYZdq{}}\PY{p}{]} \PY{o}{/} \PY{n+nb}{sum}\PY{p}{(}\PY{n}{recent\PYZus{}channel\PYZus{}dist}\PY{p}{[}\PY{l+s+s2}{\PYZdq{}}\PY{l+s+s2}{number}\PY{l+s+s2}{\PYZdq{}}\PY{p}{]}\PY{p}{)}
\PY{n}{recent\PYZus{}channel\PYZus{}dist}
\end{Verbatim}
\end{tcolorbox}

            \begin{tcolorbox}[breakable, size=fbox, boxrule=.5pt, pad at break*=1mm, opacityfill=0]
\prompt{Out}{outcolor}{8}{\boxspacing}
\begin{Verbatim}[commandchars=\\\{\}]
               number  proportion
channel
Broker            570    0.095063
Correspondent    1941    0.323716
Retail           3485    0.581221
\end{Verbatim}
\end{tcolorbox}
        
    For a more direct visualization of the distribution, below is a pie
chart constructed from the data frame above.

    \begin{tcolorbox}[breakable, size=fbox, boxrule=1pt, pad at break*=1mm,colback=cellbackground, colframe=cellborder]
\prompt{In}{incolor}{9}{\boxspacing}
\begin{Verbatim}[commandchars=\\\{\}]
\PY{c+c1}{\PYZsh{} The pie chart showing the distribution}
\PY{n}{labels} \PY{o}{=} \PY{p}{[}\PY{l+s+s2}{\PYZdq{}}\PY{l+s+s2}{Broker}\PY{l+s+s2}{\PYZdq{}}\PY{p}{,} \PY{l+s+s2}{\PYZdq{}}\PY{l+s+s2}{Correspondent}\PY{l+s+s2}{\PYZdq{}}\PY{p}{,} \PY{l+s+s2}{\PYZdq{}}\PY{l+s+s2}{Retail}\PY{l+s+s2}{\PYZdq{}}\PY{p}{]}
\PY{n}{fig1}\PY{p}{,} \PY{n}{ax1} \PY{o}{=} \PY{n}{plt}\PY{o}{.}\PY{n}{subplots}\PY{p}{(}\PY{p}{)}
\PY{n}{ax1}\PY{o}{.}\PY{n}{pie}\PY{p}{(}\PY{n}{recent\PYZus{}channel\PYZus{}dist}\PY{p}{[}\PY{l+s+s2}{\PYZdq{}}\PY{l+s+s2}{number}\PY{l+s+s2}{\PYZdq{}}\PY{p}{]}\PY{p}{,} \PY{n}{labels}\PY{o}{=}\PY{n}{labels}\PY{p}{,} \PY{n}{shadow}\PY{o}{=}\PY{k+kc}{True}\PY{p}{,} \PY{n}{startangle}\PY{o}{=}\PY{l+m+mi}{90}\PY{p}{)}
\PY{n}{ax1}\PY{o}{.}\PY{n}{axis}\PY{p}{(}\PY{l+s+s2}{\PYZdq{}}\PY{l+s+s2}{equal}\PY{l+s+s2}{\PYZdq{}}\PY{p}{)}
\PY{n}{plt}\PY{o}{.}\PY{n}{tight\PYZus{}layout}\PY{p}{(}\PY{p}{)}
\PY{n}{plt}\PY{o}{.}\PY{n}{show}\PY{p}{(}\PY{p}{)}
\end{Verbatim}
\end{tcolorbox}

    \begin{center}
    \adjustimage{max size={0.9\linewidth}{0.9\paperheight}}{report_files/report_19_0.png}
    \end{center}
    { \hspace*{\fill} \\}
    
    It is very obvious that retail initializes the majority of the home
purchases, while correspondents are also the origin of more than a
quarter of the purchases, according to the dataset from Freddie Mac.
Broker, however, only brings up less than 10\% of the purchases, making
it a minority in the origination channel distribution.

    \hypertarget{analysis-on-the-monthly-performance-dataset}{%
\section{Analysis on the Monthly Performance
Dataset}\label{analysis-on-the-monthly-performance-dataset}}

The Monthly Performance dataset represents the performance of the
mortgages being repaid after it has been initialized, on a montly basis.
It contains the information including the balance, age, modification
history, payment history and delinquency status. By interpreting the
dataset, we may be able to analyse the factors of solid performance on
payments, as well as to figure out the reasons of delinquency occurring
in mortgages.

    \begin{tcolorbox}[breakable, size=fbox, boxrule=1pt, pad at break*=1mm,colback=cellbackground, colframe=cellborder]
\prompt{In}{incolor}{10}{\boxspacing}
\begin{Verbatim}[commandchars=\\\{\}]
\PY{c+c1}{\PYZsh{} Read the monthly performance csv file}
\PY{n}{monthly\PYZus{}performance\PYZus{}data} \PY{o}{=} \PY{n}{pd}\PY{o}{.}\PY{n}{read\PYZus{}csv}\PY{p}{(}\PY{l+s+s2}{\PYZdq{}}\PY{l+s+s2}{update\PYZus{}sample.csv}\PY{l+s+s2}{\PYZdq{}}\PY{p}{,} \PY{n}{low\PYZus{}memory}\PY{o}{=}\PY{k+kc}{False}\PY{p}{)}
\end{Verbatim}
\end{tcolorbox}

    \hypertarget{loan-age-distribution-in-years-2014-to-2019}{%
\subsection{Loan Age Distribution in Years 2014 to
2019}\label{loan-age-distribution-in-years-2014-to-2019}}

In this part, I wish to see the distribution of the loan ages between
the years 2014 and 2019. Below shows the steps I took to compute this
distribution, as well as to visualize it using a histogram.

    \begin{tcolorbox}[breakable, size=fbox, boxrule=1pt, pad at break*=1mm,colback=cellbackground, colframe=cellborder]
\prompt{In}{incolor}{11}{\boxspacing}
\begin{Verbatim}[commandchars=\\\{\}]
\PY{c+c1}{\PYZsh{} Distribution of loan ages between 2014 and 2019}
\PY{n}{recent\PYZus{}performance\PYZus{}data} \PY{o}{=} \PY{n}{monthly\PYZus{}performance\PYZus{}data}\PY{o}{.}\PY{n}{loc}\PY{p}{[}
    \PY{p}{(}\PY{n}{monthly\PYZus{}performance\PYZus{}data}\PY{p}{[}\PY{l+s+s2}{\PYZdq{}}\PY{l+s+s2}{monthly\PYZus{}reporting\PYZus{}period}\PY{l+s+s2}{\PYZdq{}}\PY{p}{]} \PY{o}{/}\PY{o}{/} \PY{l+m+mi}{100} \PY{o}{\PYZgt{}}\PY{o}{=} \PY{l+m+mi}{2014}\PY{p}{)} 
    \PY{o}{\PYZam{}} \PY{p}{(}\PY{n}{monthly\PYZus{}performance\PYZus{}data}\PY{p}{[}\PY{l+s+s2}{\PYZdq{}}\PY{l+s+s2}{monthly\PYZus{}reporting\PYZus{}period}\PY{l+s+s2}{\PYZdq{}}\PY{p}{]} \PY{o}{/}\PY{o}{/} \PY{l+m+mi}{100} \PY{o}{\PYZlt{}}\PY{o}{=} \PY{l+m+mi}{2019}\PY{p}{)}
\PY{p}{]}
\PY{n}{recent\PYZus{}loan\PYZus{}age} \PY{o}{=} \PY{n}{recent\PYZus{}performance\PYZus{}data}\PY{o}{.}\PY{n}{loc}\PY{p}{[}\PY{p}{:}\PY{p}{,} \PY{l+s+s2}{\PYZdq{}}\PY{l+s+s2}{loan\PYZus{}age}\PY{l+s+s2}{\PYZdq{}}\PY{p}{]}
\PY{n}{plt}\PY{o}{.}\PY{n}{hist}\PY{p}{(}\PY{n}{recent\PYZus{}loan\PYZus{}age}\PY{p}{)}
\PY{n}{plt}\PY{o}{.}\PY{n}{xlabel}\PY{p}{(}\PY{l+s+s2}{\PYZdq{}}\PY{l+s+s2}{loan\PYZus{}age}\PY{l+s+s2}{\PYZdq{}}\PY{p}{)}
\PY{n}{plt}\PY{o}{.}\PY{n}{ylabel}\PY{p}{(}\PY{l+s+s2}{\PYZdq{}}\PY{l+s+s2}{count}\PY{l+s+s2}{\PYZdq{}}\PY{p}{)}
\PY{n}{plt}\PY{o}{.}\PY{n}{title}\PY{p}{(}\PY{l+s+s2}{\PYZdq{}}\PY{l+s+s2}{Distribution of loan\PYZus{}age between 2014 and 2019}\PY{l+s+s2}{\PYZdq{}}\PY{p}{)}
\PY{n}{plt}\PY{o}{.}\PY{n}{show}\PY{p}{(}\PY{p}{)}
\end{Verbatim}
\end{tcolorbox}

    \begin{center}
    \adjustimage{max size={0.9\linewidth}{0.9\paperheight}}{report_files/report_24_0.png}
    \end{center}
    { \hspace*{\fill} \\}
    
    By observing the histogram above, we can see a very obvious
\textbf{survival distribution} in the loan ages in the specified 6
years. The reason for it is the cumulative records for each of the
mortgages: the same mortgage will make a record each month it is in
effect, with the age incremented by a month. Over the years, old
mortgages with high loan ages are paid off or terminated, while new
mortgages appear in the dataset starting with a clean history (loan age
being 0). The fact contributes to the survival distribution as we see in
the histogram, with less records surviving in the record as the age
increases.

    \hypertarget{mortgage-fulfillment-in-2016}{%
\subsection{Mortgage Fulfillment in
2016}\label{mortgage-fulfillment-in-2016}}

At the time a mortgage is fulfilled (either by prepayment or charging
off), it is indicated on the specific record in the Monthly Performance
dataset. In specific, it will have a Zero Balance code of 1 representing
the very case of successful fulfillment of the mortgage.

In this part, I picked the year 2016 as an example, and computed the
range, as well as then mean and median, of the loan age, months to
maturity and current interest rate of the mortgages that are
successfully fulfilled in the year.

    \begin{tcolorbox}[breakable, size=fbox, boxrule=1pt, pad at break*=1mm,colback=cellbackground, colframe=cellborder]
\prompt{In}{incolor}{12}{\boxspacing}
\begin{Verbatim}[commandchars=\\\{\}]
\PY{c+c1}{\PYZsh{} Making a table of the requested values of mortgages with zero\PYZhy{}balance code}
\PY{n}{sixteen\PYZus{}performance\PYZus{}data} \PY{o}{=} \PY{n}{monthly\PYZus{}performance\PYZus{}data}\PY{o}{.}\PY{n}{loc}\PY{p}{[}
    \PY{p}{(}\PY{n}{monthly\PYZus{}performance\PYZus{}data}\PY{p}{[}\PY{l+s+s2}{\PYZdq{}}\PY{l+s+s2}{monthly\PYZus{}reporting\PYZus{}period}\PY{l+s+s2}{\PYZdq{}}\PY{p}{]} \PY{o}{/}\PY{o}{/} \PY{l+m+mi}{100} \PY{o}{==} \PY{l+m+mi}{2016}\PY{p}{)}
    \PY{o}{\PYZam{}} \PY{p}{(}\PY{n}{monthly\PYZus{}performance\PYZus{}data}\PY{p}{[}\PY{l+s+s2}{\PYZdq{}}\PY{l+s+s2}{zero\PYZus{}balance\PYZus{}code}\PY{l+s+s2}{\PYZdq{}}\PY{p}{]} \PY{o}{==} \PY{l+m+mi}{1}\PY{p}{)}
\PY{p}{]}
\PY{n}{sixteen\PYZus{}performance\PYZus{}output} \PY{o}{=} \PY{n}{pd}\PY{o}{.}\PY{n}{DataFrame}\PY{p}{(}
    \PY{n}{np}\PY{o}{.}\PY{n}{array}\PY{p}{(}\PY{p}{[}\PY{p}{[}\PY{l+m+mi}{0}\PY{p}{]} \PY{o}{*} \PY{l+m+mi}{4}\PY{p}{]} \PY{o}{*} \PY{l+m+mi}{3}\PY{p}{)}
\PY{p}{)}
\PY{n}{sixteen\PYZus{}performance\PYZus{}output} \PY{o}{=} \PY{n}{sixteen\PYZus{}performance\PYZus{}output}\PY{o}{.}\PY{n}{rename}\PY{p}{(}
    \PY{n}{index}\PY{o}{=}\PY{p}{\PYZob{}}\PY{l+m+mi}{0}\PY{p}{:} \PY{l+s+s2}{\PYZdq{}}\PY{l+s+s2}{loan\PYZus{}age}\PY{l+s+s2}{\PYZdq{}}\PY{p}{,} \PY{l+m+mi}{1}\PY{p}{:} \PY{l+s+s2}{\PYZdq{}}\PY{l+s+s2}{months\PYZus{}to\PYZus{}maturity}\PY{l+s+s2}{\PYZdq{}}\PY{p}{,} \PY{l+m+mi}{2}\PY{p}{:} \PY{l+s+s2}{\PYZdq{}}\PY{l+s+s2}{current\PYZus{}interest\PYZus{}rate}\PY{l+s+s2}{\PYZdq{}}\PY{p}{\PYZcb{}}\PY{p}{,} 
    \PY{n}{columns}\PY{o}{=}\PY{p}{\PYZob{}}\PY{l+m+mi}{0}\PY{p}{:} \PY{l+s+s2}{\PYZdq{}}\PY{l+s+s2}{mean}\PY{l+s+s2}{\PYZdq{}}\PY{p}{,} \PY{l+m+mi}{1}\PY{p}{:} \PY{l+s+s2}{\PYZdq{}}\PY{l+s+s2}{min}\PY{l+s+s2}{\PYZdq{}}\PY{p}{,} \PY{l+m+mi}{2}\PY{p}{:} \PY{l+s+s2}{\PYZdq{}}\PY{l+s+s2}{median}\PY{l+s+s2}{\PYZdq{}}\PY{p}{,} \PY{l+m+mi}{3}\PY{p}{:} \PY{l+s+s2}{\PYZdq{}}\PY{l+s+s2}{max}\PY{l+s+s2}{\PYZdq{}}\PY{p}{\PYZcb{}}
\PY{p}{)}

\PY{n}{sixteen\PYZus{}performance\PYZus{}output}\PY{p}{[}\PY{l+s+s2}{\PYZdq{}}\PY{l+s+s2}{mean}\PY{l+s+s2}{\PYZdq{}}\PY{p}{]}\PY{p}{[}\PY{l+s+s2}{\PYZdq{}}\PY{l+s+s2}{loan\PYZus{}age}\PY{l+s+s2}{\PYZdq{}}\PY{p}{]} \PY{o}{=} \PY{n}{sixteen\PYZus{}performance\PYZus{}data}\PY{p}{[}\PY{l+s+s2}{\PYZdq{}}\PY{l+s+s2}{loan\PYZus{}age}\PY{l+s+s2}{\PYZdq{}}\PY{p}{]}\PY{o}{.}\PY{n}{mean}\PY{p}{(}\PY{p}{)}
\PY{n}{sixteen\PYZus{}performance\PYZus{}output}\PY{p}{[}\PY{l+s+s2}{\PYZdq{}}\PY{l+s+s2}{min}\PY{l+s+s2}{\PYZdq{}}\PY{p}{]}\PY{p}{[}\PY{l+s+s2}{\PYZdq{}}\PY{l+s+s2}{loan\PYZus{}age}\PY{l+s+s2}{\PYZdq{}}\PY{p}{]} \PY{o}{=} \PY{n}{sixteen\PYZus{}performance\PYZus{}data}\PY{p}{[}\PY{l+s+s2}{\PYZdq{}}\PY{l+s+s2}{loan\PYZus{}age}\PY{l+s+s2}{\PYZdq{}}\PY{p}{]}\PY{o}{.}\PY{n}{min}\PY{p}{(}\PY{p}{)}
\PY{n}{sixteen\PYZus{}performance\PYZus{}output}\PY{p}{[}\PY{l+s+s2}{\PYZdq{}}\PY{l+s+s2}{median}\PY{l+s+s2}{\PYZdq{}}\PY{p}{]}\PY{p}{[}\PY{l+s+s2}{\PYZdq{}}\PY{l+s+s2}{loan\PYZus{}age}\PY{l+s+s2}{\PYZdq{}}\PY{p}{]} \PY{o}{=} \PY{n}{sixteen\PYZus{}performance\PYZus{}data}\PY{p}{[}\PY{l+s+s2}{\PYZdq{}}\PY{l+s+s2}{loan\PYZus{}age}\PY{l+s+s2}{\PYZdq{}}\PY{p}{]}\PY{o}{.}\PY{n}{median}\PY{p}{(}\PY{p}{)}
\PY{n}{sixteen\PYZus{}performance\PYZus{}output}\PY{p}{[}\PY{l+s+s2}{\PYZdq{}}\PY{l+s+s2}{max}\PY{l+s+s2}{\PYZdq{}}\PY{p}{]}\PY{p}{[}\PY{l+s+s2}{\PYZdq{}}\PY{l+s+s2}{loan\PYZus{}age}\PY{l+s+s2}{\PYZdq{}}\PY{p}{]} \PY{o}{=} \PY{n}{sixteen\PYZus{}performance\PYZus{}data}\PY{p}{[}\PY{l+s+s2}{\PYZdq{}}\PY{l+s+s2}{loan\PYZus{}age}\PY{l+s+s2}{\PYZdq{}}\PY{p}{]}\PY{o}{.}\PY{n}{max}\PY{p}{(}\PY{p}{)}

\PY{n}{sixteen\PYZus{}performance\PYZus{}output}\PY{p}{[}\PY{l+s+s2}{\PYZdq{}}\PY{l+s+s2}{mean}\PY{l+s+s2}{\PYZdq{}}\PY{p}{]}\PY{p}{[}\PY{l+s+s2}{\PYZdq{}}\PY{l+s+s2}{months\PYZus{}to\PYZus{}maturity}\PY{l+s+s2}{\PYZdq{}}\PY{p}{]} \PYZbs{}
    \PY{o}{=} \PY{n}{sixteen\PYZus{}performance\PYZus{}data}\PY{p}{[}\PY{l+s+s2}{\PYZdq{}}\PY{l+s+s2}{months\PYZus{}to\PYZus{}maturity}\PY{l+s+s2}{\PYZdq{}}\PY{p}{]}\PY{o}{.}\PY{n}{mean}\PY{p}{(}\PY{p}{)}
\PY{n}{sixteen\PYZus{}performance\PYZus{}output}\PY{p}{[}\PY{l+s+s2}{\PYZdq{}}\PY{l+s+s2}{min}\PY{l+s+s2}{\PYZdq{}}\PY{p}{]}\PY{p}{[}\PY{l+s+s2}{\PYZdq{}}\PY{l+s+s2}{months\PYZus{}to\PYZus{}maturity}\PY{l+s+s2}{\PYZdq{}}\PY{p}{]} \PYZbs{}
    \PY{o}{=} \PY{n}{sixteen\PYZus{}performance\PYZus{}data}\PY{p}{[}\PY{l+s+s2}{\PYZdq{}}\PY{l+s+s2}{months\PYZus{}to\PYZus{}maturity}\PY{l+s+s2}{\PYZdq{}}\PY{p}{]}\PY{o}{.}\PY{n}{min}\PY{p}{(}\PY{p}{)}
\PY{n}{sixteen\PYZus{}performance\PYZus{}output}\PY{p}{[}\PY{l+s+s2}{\PYZdq{}}\PY{l+s+s2}{median}\PY{l+s+s2}{\PYZdq{}}\PY{p}{]}\PY{p}{[}\PY{l+s+s2}{\PYZdq{}}\PY{l+s+s2}{months\PYZus{}to\PYZus{}maturity}\PY{l+s+s2}{\PYZdq{}}\PY{p}{]} \PYZbs{}
    \PY{o}{=} \PY{n}{sixteen\PYZus{}performance\PYZus{}data}\PY{p}{[}\PY{l+s+s2}{\PYZdq{}}\PY{l+s+s2}{months\PYZus{}to\PYZus{}maturity}\PY{l+s+s2}{\PYZdq{}}\PY{p}{]}\PY{o}{.}\PY{n}{median}\PY{p}{(}\PY{p}{)}
\PY{n}{sixteen\PYZus{}performance\PYZus{}output}\PY{p}{[}\PY{l+s+s2}{\PYZdq{}}\PY{l+s+s2}{max}\PY{l+s+s2}{\PYZdq{}}\PY{p}{]}\PY{p}{[}\PY{l+s+s2}{\PYZdq{}}\PY{l+s+s2}{months\PYZus{}to\PYZus{}maturity}\PY{l+s+s2}{\PYZdq{}}\PY{p}{]} \PYZbs{}
    \PY{o}{=} \PY{n}{sixteen\PYZus{}performance\PYZus{}data}\PY{p}{[}\PY{l+s+s2}{\PYZdq{}}\PY{l+s+s2}{months\PYZus{}to\PYZus{}maturity}\PY{l+s+s2}{\PYZdq{}}\PY{p}{]}\PY{o}{.}\PY{n}{max}\PY{p}{(}\PY{p}{)}

\PY{n}{sixteen\PYZus{}performance\PYZus{}output}\PY{p}{[}\PY{l+s+s2}{\PYZdq{}}\PY{l+s+s2}{mean}\PY{l+s+s2}{\PYZdq{}}\PY{p}{]}\PY{p}{[}\PY{l+s+s2}{\PYZdq{}}\PY{l+s+s2}{current\PYZus{}interest\PYZus{}rate}\PY{l+s+s2}{\PYZdq{}}\PY{p}{]} \PYZbs{}
    \PY{o}{=} \PY{n}{sixteen\PYZus{}performance\PYZus{}data}\PY{p}{[}\PY{l+s+s2}{\PYZdq{}}\PY{l+s+s2}{current\PYZus{}interest\PYZus{}rate}\PY{l+s+s2}{\PYZdq{}}\PY{p}{]}\PY{o}{.}\PY{n}{mean}\PY{p}{(}\PY{p}{)}
\PY{n}{sixteen\PYZus{}performance\PYZus{}output}\PY{p}{[}\PY{l+s+s2}{\PYZdq{}}\PY{l+s+s2}{min}\PY{l+s+s2}{\PYZdq{}}\PY{p}{]}\PY{p}{[}\PY{l+s+s2}{\PYZdq{}}\PY{l+s+s2}{current\PYZus{}interest\PYZus{}rate}\PY{l+s+s2}{\PYZdq{}}\PY{p}{]} \PYZbs{}
    \PY{o}{=} \PY{n}{sixteen\PYZus{}performance\PYZus{}data}\PY{p}{[}\PY{l+s+s2}{\PYZdq{}}\PY{l+s+s2}{current\PYZus{}interest\PYZus{}rate}\PY{l+s+s2}{\PYZdq{}}\PY{p}{]}\PY{o}{.}\PY{n}{min}\PY{p}{(}\PY{p}{)}
\PY{n}{sixteen\PYZus{}performance\PYZus{}output}\PY{p}{[}\PY{l+s+s2}{\PYZdq{}}\PY{l+s+s2}{median}\PY{l+s+s2}{\PYZdq{}}\PY{p}{]}\PY{p}{[}\PY{l+s+s2}{\PYZdq{}}\PY{l+s+s2}{current\PYZus{}interest\PYZus{}rate}\PY{l+s+s2}{\PYZdq{}}\PY{p}{]} \PYZbs{}
    \PY{o}{=} \PY{n}{sixteen\PYZus{}performance\PYZus{}data}\PY{p}{[}\PY{l+s+s2}{\PYZdq{}}\PY{l+s+s2}{current\PYZus{}interest\PYZus{}rate}\PY{l+s+s2}{\PYZdq{}}\PY{p}{]}\PY{o}{.}\PY{n}{median}\PY{p}{(}\PY{p}{)}
\PY{n}{sixteen\PYZus{}performance\PYZus{}output}\PY{p}{[}\PY{l+s+s2}{\PYZdq{}}\PY{l+s+s2}{max}\PY{l+s+s2}{\PYZdq{}}\PY{p}{]}\PY{p}{[}\PY{l+s+s2}{\PYZdq{}}\PY{l+s+s2}{current\PYZus{}interest\PYZus{}rate}\PY{l+s+s2}{\PYZdq{}}\PY{p}{]} \PYZbs{}
    \PY{o}{=} \PY{n}{sixteen\PYZus{}performance\PYZus{}data}\PY{p}{[}\PY{l+s+s2}{\PYZdq{}}\PY{l+s+s2}{current\PYZus{}interest\PYZus{}rate}\PY{l+s+s2}{\PYZdq{}}\PY{p}{]}\PY{o}{.}\PY{n}{max}\PY{p}{(}\PY{p}{)}

\PY{n}{sixteen\PYZus{}performance\PYZus{}output}
\end{Verbatim}
\end{tcolorbox}

            \begin{tcolorbox}[breakable, size=fbox, boxrule=.5pt, pad at break*=1mm, opacityfill=0]
\prompt{Out}{outcolor}{12}{\boxspacing}
\begin{Verbatim}[commandchars=\\\{\}]
                       mean  min  median  max
loan\_age                 57    0      47  212
months\_to\_maturity      264    0     292  477
current\_interest\_rate     4    2       4    8
\end{Verbatim}
\end{tcolorbox}
        
    Observing the statistical values of each of the variables, we can
conclude that the distribution of the loan age should be right-skewed,
which matches our observation in the previous part. The distribution of
the months to maturity, on the other hand, should be left-skewed.

Because the mean and median of the current interest rate are statiscally
equal as shown in the data frame above, I would be convinced that the
distribution of the current interst rate is more balanced. Hence, the
maximum value (8) is more likely an outlier.

    \hypertarget{factors-for-delinquent-on-payment}{%
\subsection{Factors for Delinquent on
Payment}\label{factors-for-delinquent-on-payment}}

The Monthly Performance dataset provides us with a resource to study the
general characteristics for individuals who tend to fulfill their
mortgages, as well as to recognize signs that may lead to delinquency in
paying for a mortgage. In this part, I wish to find some of the factors
that may highly likely be the sign of a future delinquency according to
the records provided in the dataset.

In the dataset, there is a column \texttt{current\_lds} that indicates
the current delinquency status of a mortgage; where provided, the value
0 stands for a current status or less than 30 days of late payment,
while any other value other than 0 stands for an over 30-day
delinquency. I benefitted from the value to categorize all records to
current mortgages and delinquent ones.

With the binary categorized data, I decided to construct a
\textbf{decision tree} model on it. The decision tree model is great in
selecting the more significant features that affect the categorization
of data; although the model is not necessarily be precisely accurate in
predicting new data, it is easy for interpretation and visualization,
and is a powerful tool for feature selection and references.

From the categorized dataset, I selected 200,000 samples to train my
decision tree model on. In specific, half of the samples (100,000) are
randomly selected from the pool of records with a current status, while
the other half are randomly selected from the pool of records with a
delinquent status. Because of the significant minority of delinquency
records than the current ones, this is the largest possible sample set I
can obtain to guarantee a similar quantity for the two target values.

Next, I randomly selected 80\% of the selected sample set to form the
training data set, while the remaining 20\% serves as the testing data
set for the model. Below is the training steps for my decision tree
model.

    \begin{tcolorbox}[breakable, size=fbox, boxrule=1pt, pad at break*=1mm,colback=cellbackground, colframe=cellborder]
\prompt{In}{incolor}{13}{\boxspacing}
\begin{Verbatim}[commandchars=\\\{\}]
\PY{c+c1}{\PYZsh{} Decision tree model}
\PY{c+c1}{\PYZsh{} Categorize the dataset to form the two targets}
\PY{n}{current\PYZus{}lds\PYZus{}data} \PY{o}{=} \PY{n}{monthly\PYZus{}performance\PYZus{}data}\PY{o}{.}\PY{n}{loc}\PY{p}{[}
    \PY{p}{(}\PY{n}{monthly\PYZus{}performance\PYZus{}data}\PY{p}{[}\PY{l+s+s2}{\PYZdq{}}\PY{l+s+s2}{current\PYZus{}lds}\PY{l+s+s2}{\PYZdq{}}\PY{p}{]} \PY{o}{!=} \PY{l+s+s2}{\PYZdq{}}\PY{l+s+s2}{XX}\PY{l+s+s2}{\PYZdq{}}\PY{p}{)}
\PY{p}{]}
\PY{n}{positive\PYZus{}data} \PY{o}{=} \PY{n}{current\PYZus{}lds\PYZus{}data}\PY{o}{.}\PY{n}{loc}\PY{p}{[}\PY{n}{current\PYZus{}lds\PYZus{}data}\PY{p}{[}\PY{l+s+s2}{\PYZdq{}}\PY{l+s+s2}{current\PYZus{}lds}\PY{l+s+s2}{\PYZdq{}}\PY{p}{]} \PY{o}{==} \PY{l+s+s2}{\PYZdq{}}\PY{l+s+s2}{0}\PY{l+s+s2}{\PYZdq{}}\PY{p}{]}
\PY{n}{negative\PYZus{}data} \PY{o}{=} \PY{n}{current\PYZus{}lds\PYZus{}data}\PY{o}{.}\PY{n}{loc}\PY{p}{[}\PY{n}{current\PYZus{}lds\PYZus{}data}\PY{p}{[}\PY{l+s+s2}{\PYZdq{}}\PY{l+s+s2}{current\PYZus{}lds}\PY{l+s+s2}{\PYZdq{}}\PY{p}{]} \PY{o}{!=} \PY{l+s+s2}{\PYZdq{}}\PY{l+s+s2}{0}\PY{l+s+s2}{\PYZdq{}}\PY{p}{]}

\PY{c+c1}{\PYZsh{} Selection of the sample data}
\PY{n}{n\PYZus{}sample} \PY{o}{=} \PY{l+m+mi}{200000}
\PY{n}{columns\PYZus{}to\PYZus{}analyse} \PY{o}{=} \PY{p}{[}
    \PY{l+s+s2}{\PYZdq{}}\PY{l+s+s2}{current\PYZus{}actual\PYZus{}upb}\PY{l+s+s2}{\PYZdq{}}\PY{p}{,} 
    \PY{l+s+s2}{\PYZdq{}}\PY{l+s+s2}{loan\PYZus{}age}\PY{l+s+s2}{\PYZdq{}}\PY{p}{,}
    \PY{l+s+s2}{\PYZdq{}}\PY{l+s+s2}{months\PYZus{}to\PYZus{}maturity}\PY{l+s+s2}{\PYZdq{}}\PY{p}{,}
    \PY{l+s+s2}{\PYZdq{}}\PY{l+s+s2}{modification\PYZus{}flag}\PY{l+s+s2}{\PYZdq{}}\PY{p}{,}
    \PY{l+s+s2}{\PYZdq{}}\PY{l+s+s2}{current\PYZus{}interest\PYZus{}rate}\PY{l+s+s2}{\PYZdq{}}\PY{p}{,}
    \PY{l+s+s2}{\PYZdq{}}\PY{l+s+s2}{current\PYZus{}deffered\PYZus{}upb}\PY{l+s+s2}{\PYZdq{}}
\PY{p}{]}
\PY{n}{sample\PYZus{}data} \PY{o}{=} \PY{n}{pd}\PY{o}{.}\PY{n}{concat}\PY{p}{(}
    \PY{p}{[}\PY{n}{positive\PYZus{}data}\PY{o}{.}\PY{n}{sample}\PY{p}{(}\PY{n}{n}\PY{o}{=}\PY{n}{n\PYZus{}sample} \PY{o}{/}\PY{o}{/} \PY{l+m+mi}{2}\PY{p}{)}\PY{p}{,} \PY{n}{negative\PYZus{}data}\PY{o}{.}\PY{n}{sample}\PY{p}{(}\PY{n}{n}\PY{o}{=}\PY{p}{(}\PY{n}{n\PYZus{}sample} \PY{o}{\PYZhy{}} \PY{n}{n\PYZus{}sample} \PY{o}{/}\PY{o}{/} \PY{l+m+mi}{2}\PY{p}{)}\PY{p}{)}\PY{p}{]}
\PY{p}{)}\PY{o}{.}\PY{n}{sample}\PY{p}{(}\PY{n}{frac}\PY{o}{=}\PY{l+m+mi}{1}\PY{p}{)}
\PY{n}{sample\PYZus{}data} \PY{o}{=} \PY{n}{sample\PYZus{}data}\PY{o}{.}\PY{n}{replace}\PY{p}{(}\PY{p}{\PYZob{}}\PY{l+s+s2}{\PYZdq{}}\PY{l+s+s2}{modification\PYZus{}flag}\PY{l+s+s2}{\PYZdq{}}\PY{p}{:} \PY{p}{\PYZob{}}\PY{l+s+s2}{\PYZdq{}}\PY{l+s+s2}{Y}\PY{l+s+s2}{\PYZdq{}}\PY{p}{:} \PY{l+m+mi}{1}\PY{p}{,} \PY{n}{np}\PY{o}{.}\PY{n}{nan}\PY{p}{:} \PY{l+m+mi}{0}\PY{p}{\PYZcb{}}\PY{p}{\PYZcb{}}\PY{p}{)}

\PY{c+c1}{\PYZsh{} Select the training and test data sets}
\PY{n}{training\PYZus{}data}\PY{p}{,} \PY{n}{test\PYZus{}data} \PY{o}{=} \PY{n}{sample\PYZus{}data}\PY{p}{[}\PY{p}{:}\PY{n+nb}{int}\PY{p}{(}\PY{l+m+mf}{0.8} \PY{o}{*} \PY{n}{n\PYZus{}sample}\PY{p}{)}\PY{p}{]}\PY{p}{,} \PY{n}{sample\PYZus{}data}\PY{p}{[}\PY{n+nb}{int}\PY{p}{(}\PY{l+m+mf}{0.8} \PY{o}{*} \PY{n}{n\PYZus{}sample}\PY{p}{)}\PY{p}{:}\PY{p}{]}
\PY{n}{y\PYZus{}train}\PY{p}{,} \PY{n}{x\PYZus{}train} \PY{o}{=} \PY{n}{np}\PY{o}{.}\PY{n}{array}\PY{p}{(}\PY{n}{training\PYZus{}data}\PY{p}{[}\PY{l+s+s2}{\PYZdq{}}\PY{l+s+s2}{current\PYZus{}lds}\PY{l+s+s2}{\PYZdq{}}\PY{p}{]} \PY{o}{==} \PY{l+s+s2}{\PYZdq{}}\PY{l+s+s2}{0}\PY{l+s+s2}{\PYZdq{}}\PY{p}{)}\PY{p}{,} \PYZbs{}
    \PY{n}{training\PYZus{}data}\PY{o}{.}\PY{n}{loc}\PY{p}{[}\PY{p}{:}\PY{p}{,} \PY{n}{columns\PYZus{}to\PYZus{}analyse}\PY{p}{]}
\PY{n}{y\PYZus{}test}\PY{p}{,} \PY{n}{x\PYZus{}test} \PY{o}{=} \PY{n}{np}\PY{o}{.}\PY{n}{array}\PY{p}{(}\PY{n}{test\PYZus{}data}\PY{p}{[}\PY{l+s+s2}{\PYZdq{}}\PY{l+s+s2}{current\PYZus{}lds}\PY{l+s+s2}{\PYZdq{}}\PY{p}{]} \PY{o}{==} \PY{l+s+s2}{\PYZdq{}}\PY{l+s+s2}{0}\PY{l+s+s2}{\PYZdq{}}\PY{p}{)}\PY{p}{,} \PYZbs{}
    \PY{n}{test\PYZus{}data}\PY{o}{.}\PY{n}{loc}\PY{p}{[}\PY{p}{:}\PY{p}{,} \PY{n}{columns\PYZus{}to\PYZus{}analyse}\PY{p}{]}

\PY{c+c1}{\PYZsh{} Fit the training data to the decision tree}
\PY{n}{dt} \PY{o}{=} \PY{n}{DecisionTreeClassifier}\PY{p}{(}\PY{n}{criterion}\PY{o}{=}\PY{l+s+s2}{\PYZdq{}}\PY{l+s+s2}{gini}\PY{l+s+s2}{\PYZdq{}}\PY{p}{)}
\PY{n}{dt}\PY{o}{.}\PY{n}{fit}\PY{p}{(}\PY{n}{x\PYZus{}train}\PY{p}{,} \PY{n}{y\PYZus{}train}\PY{p}{)}
\end{Verbatim}
\end{tcolorbox}

            \begin{tcolorbox}[breakable, size=fbox, boxrule=.5pt, pad at break*=1mm, opacityfill=0]
\prompt{Out}{outcolor}{13}{\boxspacing}
\begin{Verbatim}[commandchars=\\\{\}]
DecisionTreeClassifier(ccp\_alpha=0.0, class\_weight=None, criterion='gini',
                       max\_depth=None, max\_features=None, max\_leaf\_nodes=None,
                       min\_impurity\_decrease=0.0, min\_impurity\_split=None,
                       min\_samples\_leaf=1, min\_samples\_split=2,
                       min\_weight\_fraction\_leaf=0.0, presort='deprecated',
                       random\_state=None, splitter='best')
\end{Verbatim}
\end{tcolorbox}
        
    After training the decision tree model over my training data set, I used
the model to predict my test data set, and evaluates its performance by
checking its hit rate.

    \begin{tcolorbox}[breakable, size=fbox, boxrule=1pt, pad at break*=1mm,colback=cellbackground, colframe=cellborder]
\prompt{In}{incolor}{14}{\boxspacing}
\begin{Verbatim}[commandchars=\\\{\}]
\PY{c+c1}{\PYZsh{} Predict the test data using the trained decision tree}
\PY{n}{predict\PYZus{}output} \PY{o}{=} \PY{n}{dt}\PY{o}{.}\PY{n}{predict}\PY{p}{(}\PY{n}{x\PYZus{}test}\PY{p}{)}

\PY{c+c1}{\PYZsh{} Compute the accuracy rate for the prediction}
\PY{n}{corr} \PY{o}{=} \PY{l+m+mi}{0}
\PY{k}{for} \PY{n}{i} \PY{o+ow}{in} \PY{n+nb}{range}\PY{p}{(}\PY{l+m+mi}{0}\PY{p}{,} \PY{n+nb}{len}\PY{p}{(}\PY{n}{predict\PYZus{}output}\PY{p}{)}\PY{p}{)}\PY{p}{:}
    \PY{k}{if} \PY{n}{predict\PYZus{}output}\PY{p}{[}\PY{n}{i}\PY{p}{]} \PY{o}{==} \PY{n}{y\PYZus{}test}\PY{p}{[}\PY{n}{i}\PY{p}{]}\PY{p}{:}
        \PY{n}{corr} \PY{o}{+}\PY{o}{=} \PY{l+m+mi}{1}
\PY{n}{acc\PYZus{}rate} \PY{o}{=} \PY{n}{corr} \PY{o}{/} \PY{n+nb}{len}\PY{p}{(}\PY{n}{predict\PYZus{}output}\PY{p}{)}
\PY{n}{acc\PYZus{}rate}
\end{Verbatim}
\end{tcolorbox}

            \begin{tcolorbox}[breakable, size=fbox, boxrule=.5pt, pad at break*=1mm, opacityfill=0]
\prompt{Out}{outcolor}{14}{\boxspacing}
\begin{Verbatim}[commandchars=\\\{\}]
0.824675
\end{Verbatim}
\end{tcolorbox}
        
    As shown above, the model achieved \(\geq\) 80\% accuracy when
predicting over the test data set. This provides me with the confidence
in the model, that the features selected by it should be reasonable and
significant for predicting the delinquency status of a mortgage.

With this is confirmed, I visualize the decision tree in a tree graph.
Considering the vast branching factor of the model, I visualized only
the top 3 levels of the tree, which displays the most significant
factors in the prediction. The graph is demonstrated as seen below.

    \begin{tcolorbox}[breakable, size=fbox, boxrule=1pt, pad at break*=1mm,colback=cellbackground, colframe=cellborder]
\prompt{In}{incolor}{15}{\boxspacing}
\begin{Verbatim}[commandchars=\\\{\}]
\PY{n}{dot\PYZus{}data} \PY{o}{=} \PY{n}{StringIO}\PY{p}{(}\PY{p}{)}
\PY{n}{export\PYZus{}graphviz}\PY{p}{(}\PY{n}{dt}\PY{p}{,} 
                \PY{n}{out\PYZus{}file}\PY{o}{=}\PY{n}{dot\PYZus{}data}\PY{p}{,} 
                \PY{n}{max\PYZus{}depth}\PY{o}{=}\PY{l+m+mi}{3}\PY{p}{,} 
                \PY{n}{filled}\PY{o}{=}\PY{k+kc}{True}\PY{p}{,} 
                \PY{n}{rounded}\PY{o}{=}\PY{k+kc}{True}\PY{p}{,}
                \PY{n}{feature\PYZus{}names}\PY{o}{=}\PY{n}{columns\PYZus{}to\PYZus{}analyse}\PY{p}{,}
                \PY{n}{special\PYZus{}characters}\PY{o}{=}\PY{k+kc}{True}\PY{p}{)}
\PY{n}{graph} \PY{o}{=} \PY{n}{pydotplus}\PY{o}{.}\PY{n}{graph\PYZus{}from\PYZus{}dot\PYZus{}data}\PY{p}{(}\PY{n}{dot\PYZus{}data}\PY{o}{.}\PY{n}{getvalue}\PY{p}{(}\PY{p}{)}\PY{p}{)}  
\PY{n}{Image}\PY{p}{(}\PY{n}{graph}\PY{o}{.}\PY{n}{create\PYZus{}jpg}\PY{p}{(}\PY{p}{)}\PY{p}{)}
\end{Verbatim}
\end{tcolorbox}
 
            
\prompt{Out}{outcolor}{15}{}
    
    \begin{center}
    \adjustimage{max size={0.9\linewidth}{0.9\paperheight}}{report_files/report_34_0.jpeg}
    \end{center}
    { \hspace*{\fill} \\}
    

    By observing the tree graph above, we can see that records with a
current status tends to have, in common, lower interest rate; however,
if the interest is too low (in specific, lower than 2.5\%), there occurs
a strong tendency for the mortgage record to develop a delinquency.
Current mortgage records also tend to have fewer months to maturity,
while records with more than 30 years to maturity are more likely to be
delinquent. In addition, records with a delinquent status also tend to
have a high interest rate in common.

From the observation, we can conclude that the risk of delinquency in
mortgage payment is higher when:

\begin{itemize}
\tightlist
\item
  Current interest rate is beyond the 2.5\%\textasciitilde5.5\% range,
  or
\item
  Time to legal maturity is over 30 years.
\end{itemize}

Again, the conclusions are for reference use only, as decision tree is
not regarded as a precise predicting model. However, the feature
selection will be useful for conducting further analysis on the problem
using more precise techniques like linear regression analysis.


    % Add a bibliography block to the postdoc
    
    
    
\end{document}
